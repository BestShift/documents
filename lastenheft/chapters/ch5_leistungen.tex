\newpage
\chapter{Produktleistungen}

\section{/LL10/ Kompatibilität}
\newline
Das Produkt soll zur aktuellen Android-Version 4.4 bzw. 5.0 kompatibel sein und mit allen Fahrzeugen mit ODB2-Diagnoseport funktionieren. Es ist möglich, dass nicht alle Funktionen des Systems in allen Fahrzeugen zur Verfügung stehen.

\section{/LL20/ Hands Free}
\newline
Da BestShift während des Autofahrens verwendet soll ist es besonders wichtig dass die Applikation komplett ohne Input des Fahrers funktioniert und sich daher alle benötigten Daten aus dem, mittels Bluetooth gekoppelten CarPC, holt. 

\section{/LL30/ simple GUI}
\newline
Da die Informationen der App oder auch des Widgets zu jedem Zeitpunkt auf einen Blick für den Fahrer erkennbar sein sollen, ist der einfache Aufbau der GUI besonders wichtig. Gute Ablesbarkeit der gezeigten Informationen, angepasst auf übliche Abstände zwischen Fahrer und Geräteständer in Fahrzeugen, sowie große Bedienelemente müssen vorgesehen werden.

\section{/LL40/ Exportfunktion für eigene Verbrauchs- und Komfortanalyse zu Hause}
\newline
Es soll eine Exportfunktion geben um die gespeicherten Daten und Informationen zu speichern oder aus dem CarPC zu entfernen und diese dann analysieren zu können. Hierfür dürfen nur offene Fileformate verwendet werden um dem Benutzer die Weiterverarbeitung z.B. über eine Tabellenkalkulation zu erleichtern. 

\section{/LL50/ Preis}
\newline
Der Endkundenpreis für die notwendige Hardware sollte € 250,- nicht überschreiten.