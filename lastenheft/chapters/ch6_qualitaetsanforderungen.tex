\newline
\chapter{Qualitätsanforderungen}
\begin{table}[!htb]
\centering
\caption{Qualitätsanforderungen Tabelle}
\label{q-anforderungen}
\begin{tabular}{|l|c|c|c|c|}
\hline
Produktqualität & Sehr gut &   Gut   & Normal & Irrelevant \\ \hline
Funktionalität & x &  &  &  \\ \hline
Zuverlässigkeit & x &  &  &  \\ \hline
Benutzbarkeit & x &  &  &  \\ \hline
Effizienz &  &  &  & x \\ \hline
Änderbarkeit &  &  &  & x \\ \hline
Übertragbarkeit &  &  & x &  \\ \hline
\end{tabular}
\end{table}

\section{Funktionalität und Zuverlässigkeit}
Diese Anforderungen sind besonders wichtig, da Abstürze oder Fehlermeldungen den Fahrer vom Straßenverkehr ablenken könnte.

\section{Benutzbarkeit}
Die GUI muss möglichst übersichtlich gestaltet werden, und an eine Bedienung während der Fahrt angepasst werden.

\section{Effizienz und Änderbarkeit}
Die Effizienz und Änderbarkeit können irrelevant sein, weil die Änderbarkeit sich nur auf den CarPC beschränkt, aber der User nur die Modi wechseln kann, was nicht als Änderung anzusehen ist und die Effizienz kann aufgrund des Vorhandenseins eines Ladegeräts in nahezu jedem Auto als irrelevant gesehen werden. Einzig eine mögliche Überhitzung muss bei einigen Mobiltelefonen als Risiko betrachtet werden.

\section{Übertragbarkeit}
Das Tool ist nicht auf jedes mögliche Fahrzeug übertragbar. Grundvoraussetzung ist dass das Auto eine OBD2 Schnittstelle hat. Im Projektumfang wird es voraussichtlich nur mit wenigen Fahrzeugen getestet werden können, wodurch eine generelle Übertragbarkeit hier nicht gewährleistet werden kann.