\chapter{Produktfunktionen}

\section{/LF10/ Hardware und Sensorik}
\textbf{Geschäftsprozess:}	Verarbeitung der Sensor Daten
\newline
\textbf{Akteur:}		CarPC
\newline
\textbf{Beschreibung:}	Aufbau von Sensoren zur Erfassung von Fahrdaten und Ansteuerung von Schnittstellen zum Motormanagement in einem einfach zu installierenden CarPC.

\section{/LF11/ CarPC}
Es soll ein Single Board Computer (SBC) verwendet werden, welcher alle verwendeten Sensoren unterstützt. Die Montage darf den Fahrer nicht behindern. Idealerweise ist der CarPC mobil auszuführen, damit er in mehreren Fahrzeugen verwendet werden kann. Die Stromversorgung muss über das 12V Bordnetz eines Kfz möglich sein, bei Fixeinbau muss diese mit der Zündung gekoppelt werden.

\section{/LF12/ Anbindung der OBD Schnittstelle an den CarPC}
Der CarPC soll Motordaten aus der standardisierten Diagnoseschnittstelle des Fahrzeugen auslesen können. Diese Motordaten umfassen beispielsweise die Drehzahl des Motors, die Fahrgeschwindigkeit und wenn möglich auch den eingelegten Gang.

\section{/LF13/ Messung der Fahrzeugbeschleunigung in Fahrzeuglängst- und -querachse}
Die Beschleunigung in Fahrzeuglängs- (Beschleunigung und Bremsen) sowie die Fahrzeugquerachse (Kurvenbeschleunigung) sollen in die Verbrauchsinformation und Fahrgastbequemlichkeit einfließen.

\section{/LF14/ Messung der Fahrzeugneigung in 3 Raumachsen}
Die Drehung des Fahrzeug in alle Raumachsen wird mit dem Gyroskop gemessen. Die Bewegung um die Raumachsen heißen bei einem Kfz Kippen, Rollen und Gieren. Diese Informationen werden in weiterer Folge vor allem für die Erkennung eines Hangs eingesetzt, aber auch für die Fahrgastbequemlichkeit.

\section{/LF15/ Messen der geographischen Fahrzeugposition}
Ein GPS Sensor muss am CarPC angeschlossen sein um den Standort des Fahrzeugs ermitteln zu können und um die Verbrauchs- und Neigungswerte kartieren zu können.

\section{/LF16/Messung des Fahrgastraumklimas}
Ein Temperatursensor und mögliche weitere Sensoren werden im CarPC integriert um die Fahrgastbequemlichkeit besser beurteilen zu können.

\section{/LF17/ Multifahrzeug-Managment}
Der Benutzer soll die Möglichkeit haben, ein Fahrzeugprofil auszuwählen (in Verbindung mit der Android-App). Damit soll eine mehrfache Verwendung des CarPCs (so eine mobile Lösung implementiert wurde) in verschiedenen Fahrzeugen gewährleistet werden.

\newpage

\section{/LF20/ Datenmanagement und Datenanalyse}
\textbf{Geschäftsprozess:}	Verarbeitung der Sensor Daten
\newline
\textbf{Akteur:}		CarPC, standardisierte informationsschnittstelle
\newline
\textbf{Beschreibung:} Aufbereitung und effiziente Speicherung von Daten verschiedener Fahrzeugsensoren und Implementierung von Schnittstellen für weitere Applikationen. 

\section{/LF21/ Sammlung und Speicherung aller Sensordaten und Ablegen in einer strukturierten Form}
Die Sensordaten müssen gesammelt und so gespeichert werden, dass sie strukturiert und zur Weiterverarbeitung vorbereitet sind.

\section{/LF22/ Interpretation der Rohdaten in Messgrößen mit gängigen Einheiten}
Alle Rohdaten müssen in einen normierten Wert in einer sinnvollen Einheit umgewandelt werden. Dabei ist besonders zu beachten, ob und wie die Positionierung der Sensoren im Fahrzeug auf die Messwerte einen Einfluss hat.

\section{/LF23/ Filterung von offensichtlichen Messfehlern}
Bei dieser Funktion ist vor allem die Auswahl was als Messfehler gewertet wird entscheidend. Denn es sollen zwar Messfehler gefunden und entfernt werden, allerdings sollen auch die Maxima der wirklichen Fahrt nicht fälschlicher Weise gefiltert werden. Daher hängt diese Filterung auch sehr nahe mit der Interpretation der Rohdaten zusammen.

\section{/LF24/ Zusammenfassung von Messgrößen}
Zur einfacheren Weiterverarbeitung sollen Messwerte (wo sinnvoll) schon vom CarPC sinnvoll vorverarbeitet und zusammengefasst werden.

\section{/LF25/ Implementierung einer Schnittstelle zur Kommunikation mit der Android App}
Die auf dem CarPC gesammelten Daten müssen sich einfach von der Android-App (siehe LF30/40/50) abrufen lassen. Dafür muss eine Schnittstelle entwickelt werden, die energieeffiezient arbeitet, aber auch schnell genug ist, Momentanwerte zeitgerecht zu übertragen.