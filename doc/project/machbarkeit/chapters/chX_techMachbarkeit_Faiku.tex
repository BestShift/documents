\chapter{Appentwicklung}

\section{Verbrauchsanalyse}

	\subsection{Vergleich Java und Python}
		\subsubsection{Java}
			\begin{itemize}
				\item Charakteristik: Mehrfachverwendung/Mehrfachvererbung eingeschränkt, stark Typisiert
				\item Syntax: Vor der Kompilierung sehr streng, komplizierte Polymorphie
				\item Dokumentation: Ausführliche Dokumentation der Methoden in Form von Java API
				\item Community: Java ist sehr weit verbreitet, rund 3 Milliarden Geräte laufen mit Java.
				\item Usage: plattformunabhängige Programme, FAT-Client-Anwendungen
				\item Können im Team: 5/5*, wird seit der 1.Klasse im TGM Unterrichtet.
			\end{itemize}

		\subsubsection{Python}
			\begin{itemize}
				\item Charakteristik: Erstellung von Variablen nicht Datentyp abhängig, dynamisch Typisiert.
				\item Syntax: Einrückung größtes Syntaktisches Problem, daher Geschmackssache im Vergleich zu Java
				\item Dokumentation: Vorhanden, aber nicht ausführlich.
				\item Community: Die Community von Python ist ständig am wachsen, ca. 5 Millionen Geräte arbeiten mit Python.
				\item Usage: Back-End, äußerst schnelle Entwicklung(Rapid Application Development),
				\item Können im Team: 3/5*, wird ab der 5.Klasse im TGM Unterrichtet.
			\end{itemize}
			
			Aus den Oben genannten Punkten filtern wir heraus, dass Java die bessere Lösung wäre.
			Die komplette Beherrschung von Python würde einige Wochen dauern. Da diese Zeit im Rahmen
			des Diplomprojektes nicht gegeben ist, wäre die App-Entwicklung in Python nicht zielführend.
			Außerdem benutzen Android-Smartphones die Dalvik-VM, welche Java Class Files exekutiert.
			Dementsprechend wird hier massiv an Umkompilierungszeit erspart, welche für die Echtzeit-Darstellung
			des Verbrauchs, Kalmans und Schaltvorgangs essentiell ist.
\subsection{Frameworks}
\begin{table}[]
\centering
\caption{My caption}
\label{my-label}
\begin{tabular}{lllll}
 & MPAndroidChart & AChartEngine & HelloChart & Erklärung \\
Performance & gut & gut & gut &  \\
Supported laguages & IOS 7,8 , App java & java & java & MPAC kann auch weiterverwendet werden für IOS apps \\
Documentation & 85% & 70% & 50% & MPAC hat eine sehr gut in kapiteln aufgelistete Dokumentation, Tutorials \\
Recent activity & 13 tage zuvor & ein ein halb Jahre zuvor & 6Tage zuvor & MPAC ist sehr aktuell und bringt immer wieder neue bugfixes raus. Hello Chart genau so \\
Community & 80% & 60% & 20% & MPAC hat eine sehr übersichtliche Community im Gegensatz zu ACHartEngine. Eine Antwort wird innerhalb von 7 tagen geliefert. HelloCHart hat eine sehr alte bis keine Community \\
Usage examples & 85% & 70% & 15% & MPAC hat für jeden Chart ein eigenes Beispiel oder auch Tutorials in Youtube. AChartEngine hat eher mehr Tutorials in Youtube und nicht so viel greifbaren Code wie MPAC.
\end{tabular}
\end{table}

Im Großen und Ganzen ist MPAndroidChart das Beste was man wählen kann, da es eins der besten Dokumentationen hat, weiterhin hat es auch eine sehr gute Community, sehr viel gut gecodete und Dokumentierte Beispiele und ist auch für Erweiterung gut , da die App auch auf einem IPhone programmiert werden kann. 
AChartEngine hat eine gute Dokumentation doch eine eher schwächere Community und ist leider nicht sehr aktuell. 
HelloChart hat eins der schlechtesten Dokumentationen, gar keine Community und sehr wenig gecodete Beispiele, sie sind aber sehr Aktuell wenn es um die Versionen geht .
 
