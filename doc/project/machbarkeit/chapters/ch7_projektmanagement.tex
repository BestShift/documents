\chapter{Projektmanagement}

Prinzipiell stehen zwei Angehensweisen zur Verfügung, da wir nur mit diesen bereits gearbeitet haben. Die beiden sind: agiles oder traditionelles Projektmanagement.

\section{Traditionelles Projektmanagement}
Bei der traditionellen Variante wird jede Phase als eine Sequenz von Arbeitsstritten, welche fertig gestellt werden müssen, definiert. Die einzelnen Phasen werden wie folgt festgelegt:
\begin{itemize}
		\item Initialisierung
		\item Planung und Designüberlegung
		\item Erstellung
		\item Monitoring und Kontrollsysteme
		\item Fertigstellung
\end{itemize}

Ein großer Vorteil bei dieser Methodik ist, dass man ganz genau weiß was man als nächstes zu tun ist. Dies kann aber auch ein riesiger Nachteil sein, wenn man die notwendigen Dokumente nicht ausführlich genug geschrieben hat. Hierbei verliert man dann Zeit und Resourcen.

\todo Pro und Contra noch finden
\subsection{Pro Argumente}
\begin{itemize}
		\item Potentielle Probleme welche während der Developmentphase gefunden werden, können schon in der Designphase recherchiert werden.
		\item 
\end{itemize}

\subsection{Contra Argumente}


\section{Agiles Projektmanagement}
Unter agilem Projektmanagement, insbesondere Scrum, gibt es den sogenannten Sprint; ein zeitlich Begrenzter Arbeitsaufwand. Ein Spirnt wird logischerweise im Vorhinein geplant, wobei man Wert auf den \"scope of work\" (zu erledigende Arbeitspakete) legt. Zusätzlich dazu, gibt es zum Beginn einer jeden Arbeitsphase ein kleines Meeting (Daily Scrum Meeting). In diesem Meeting erklärt jedes Mitglied was es in der letzten Arbeitsphase gearbeitet hat, was es in dieser Arbeitsphase tun wird und welche Probleme es beim Arbeiten gegeben hat.

Anders als bei der traditionellen Variante hat Scrum den großen Vorteil das neue User Stories, und damit euch neue Funktionen der Software, on-the-go hinzugefügt werden können. Grundsätzlich kann man sagen, dass Scrum Sinnvoll ist, wenn man mögliche Veränderungen anstrebt.

\subsection{Pro Argumente}
\begin{itemize}
		\item Kommunikation zwischen Kunden und Developern muss immer da sein
		\item Kontinuierliches feedback
		\item Flexibler zu Veränderungen
		\item Klarheit über potentielle Problemquellen
\end{itemize}

\subsection{Contra Argumente}
\begin{itemize}
		\item Das team muss Scrum verstehen und es anwenden, viel kommunikation notwendig
		\item Es ist leicht die disziplin zu verlieren
		\item Team member Anzahl sollte weniger als 6 Leute sein
\end{itemize}


\section{Conclusio}
Durch die gegebenen Zustände entschieden wir uns für eine agile Projektmanagement Methode, um genau zu sein Scrum. Da wir auch für Veränderungen offen sein wollen ist dies eine gute Gelegenheit. Der Projektbetreuer, Erhard List, ist ein Befürworter für diese Angehensweise. 
