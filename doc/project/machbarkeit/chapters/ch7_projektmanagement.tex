\chapter{Projektmanagement}

Prinzipiell standen zwei Angehensweisen zur Verfügung, da wir nur mit zweien bereits gearbeitet haben. Die beiden sind: agiles oder traditionelles Projektmanagement.
In den folgenden Zeilen werden die beiden etwas detaillierter Beschrieben.

\section{Traditionelles Projektmanagement}
Bei der traditionellen Variante wird jede Phase als eine Sequenz von Arbeitsstritten, welche fertig gestellt werden müssen, definiert. Die einzelnen Phasen werden wie folgt definiert:
\begin{itemize}
		\item Initialisierung
		\item Planung und Designüberlegung
		\item Erstellung
		\item Monitoring und Kontrollsysteme
		\item Fertigstellung
\end{itemize}

Ein großer Vorteil bei dieser Methodik ist, dass man ganz genau weiß was man als nächstes zu tun ist. Dies kann aber auch ein riesiger Nachteil sein, wenn man sich nicht sicher ist, was man genau will.

\subsection{Pros}
\begin{itemize}
		\item Potentielle Probleme welche während der Developmentphase gefunden werden, können schon in der Designphase recherchiert werden.
		\item 
\end{itemize}

\subsection{Cons}

\newpage

\section{Agiles Projektmanagement}
Unter agilem Projektmanagement, insbesondere Scrum, gibt es den sogenannten Sprint; ein zeitlich Begrenzter Arbeitsaufwand. Dieser Sprint wird logischerweise davor geplant, hierbei legt man Wert auf den „Scope of work“, die Teilprodukte die in diesen Spring geschafft werden können, uvm. Zusätzlich dazu gibt zu beginn einer jeden Arbeitsphase an einem Tag, ein kleines Meeting (Daily Scrum) in dem das gesamte Team Mitglied erklären muss was er gestern gemacht hat und in wie fern es zum Sprint beigetragen hat; was man heute vor hat und in wie fern es zum Sprint beiträgt; und ob man irgendwelche Probleme hat, bzw. haben könnte, welche das Ziel des Sprints verhindern könnten.

Anders als bei der traditionellen Variante hat Scrum den großen Vorteil das neue User Stories, und damit euch neue Funktionen der Software, on-the-go hinzugefügt werden. Grundsätzlich kann man sagen, dass Scrum Sinnvoll ist, wenn man nicht sich nicht sicher über das Endprodukt ist.

\subsection{Pros}
\begin{itemize}
		\item Kommunikation zwischen Kunden und Developern muss immer da sein
		\item Kontinuierliches feedback
		\item Flexibler zu Veränderungen
		\item Klarheit von wo probleme kommen
\end{itemize}

\subsection{Cons}
\begin{itemize}
		\item Das team muss Scrum verstehen und es anwenden, viel kommunikation notwendig
		\item Es ist leicht die disziplin zu verlieren
		\item Team member Anzahl sollte weniger als 6 Leute sein
\end{itemize}



\section{Conclusio}
Da wir uns nicht sicher sind, wie das Endprodukt sein wird, ebenso wenig wie die Umsetzung und passenden Technologien aussehen, habe wir uns für Scrum entschieden. Nach Absprache mit dem Projektbetreuer, Erhard List, wurde dies auch offiziell festgelegt.
