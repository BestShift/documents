\chapter{Voruntersuchung}

\section{Ist-Zustand}
Der momentane Zustand von Hilfsmodulen für FahranfängerInnen, die nachhaltig ihr Fahrverhalten verbessern möchten, hat keine große Bandbreite. Für KFZ-Neulinge ist es besonders schwer, ein richtiges Fahrgefühl zu entwickeln, da die meisten Fahrschulen den Informationsfluss auf die mündliche Weitergabe beschränken. Da die Informationen durch den Fahrlehrer weitergegeben werden und nicht immer direkt auf den Fahrschüler eingegangen werden kann, können hier grundsätzliche Fehler entstehen. Der Lernprozess der korrekten und dauerhaften Inbetriebnahme eines KFZ ist äußerst komplex. Ebenfalls ist das Übertragen von Erfahrung auf andere Fahrer nur eingeschränkt möglich.
Da ein Neuwagen mit ausgeprägtem Bord-Computer von vielen Fahranfängern finanziel schwer leistbar ist und ältere Fahrzeugmodelle diese Verbesserungen des Fahrverhaltens nicht anbieten, ist dieses Produkt besonders für Fahranfänger interessant.

\section{Soll-Zustand}
Die Software soll das erlernen und optimieren des Fahrstils des Kunden erleichtern. Durch einen Algorithmus zur Berechnung des optimalen Gangwechsels soll das Prinzip eines manuellen Getriebes und auch die Drehzahlbereiche der eigenen Gangschaltung besser verstanden werden. Die zum Gang passende Drehzahl soll gefunden werden und akustisch und/oder visuell dargestellt werden. Die momentanen Fahreigenschaften (Fahrgastbequemlichkeit und Verbrauch) sollen ebenfalls angezeigt werden. \newpage Nach einer Fahrt werden, mithilfe der gesammelten Daten, Graphen erstellt und die Eigenschaften der Fahrt werden mittels eines Overlays auf eine Karte gelegt. Die Maxima der Daten werden optisch hervorgehoben. Diese können dann Retrospektiv zu Lernzwecken verwendet werden. Die Fahrgastbequemlichkeit soll in Form von einem Kamm'schen Kreis dargestellt werden, um die momentanen Querbeschleunigungen zu visualisieren, während der Kraftstoffverbrauch mit berechneten Werten angezeigt ist.

\section{Ziele}
\subsection{Einfaches Softwarekonzept}
Die Android App ist einfach zu bedienen sein. Zusätzlich hat die grafische Oberfläche wenige Eingabefelder um Fehler zu vermeiden.
Da die graphische Oberfläche beim Fahren nicht stören darf, sollen die vorhandenen Eingabefelder möglichst groß und leicht zu bedienen sein. Durch akustische Meldungen soll der Blick des Fahrers nicht von der Fahrbahn abgeleitet werden.

\subsection{Abgrenzung von möglichen Fehlerquellen}
BestShift ist ausdrücklich so entwickelt, dass die Motordaten aus der OBD-II Schnittstelle direkt in eine Datenbank übertragen werden, um sie für die App einbindbar zu machen. Der Vorteil davon ist, dass das System so eine gute Kapselung ausweist.

\subsection{Dokumentation}
Im Laufe des Projektes werden folgende Sachen mit höchster Priorität dokumentiert:
\begin{itemize}
\item Zeitaufzeichnung
\item Codekommentare
\item Testing
\end{itemize}


\todo Mehr ziele?


\section{Nicht-Ziele}

\subsection{Portabilität/Limitierung}
Die Software wird vorerst nicht für alle Automobil-Marken verfügbar sein. Das Konzept besteht für eine gewisse Anzahl von Automobilen, die im Laufe des Projektes zur Realisierung verwendet werden können. Da wir keine Fahrzeuge direkt von einem Hersteller zu Verfügung gestellt bekommen, können wir nur mit den Fahrzeugen aus unserem direkten Kreis arbeiten. 

\subsection{Netzwerkabhängigkeit}
Nicht-Ziel von BestShift ist es, die Daten in die Cloud zu laden um sie dann in einer Webapplikation analysieren zu können. Die Daten, die aus dem Motormanagement entnommen worden sind, können auf den Server synchronisiert  werden. 

\subsection{Direkte Beeinflussung des Fahrverhaltens}
Wir distanzieren uns stets von dem Gedanken, dass unser Produkt das Fahrverhalten eines Kunden direkt beeinflusst.
Es werden lediglich Vorschläge gegeben, die zu den gegebenen Zuständen und Informationen,
die aus dem Motormanagement entnommen werden, angepasst sind.


\todo Mehr nicht ziele?