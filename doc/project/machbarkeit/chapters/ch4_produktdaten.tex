\chapter{Produktdaten}

Die Produktdaten sind essenziell für das Produkt, da alle Daten benötigt werden, um die Applikation vollständig verwenden zu können. Auch ein Verbindungsproblem zur OBD-II Schnittstelle würde große Probleme erzeugen, allerdings wäre die App trotzdem noch, aufgrund der vom CarPC zur Verfügung gestellten Daten, beschränkt einsetzbar.

Die Daten sollten auf dem CarPC gespeichert werden, beziehungsweise bei einem Shutdown des CarPC bekommt der Anwender die möglichkeit die neuen Informationen in die Cloud der Webapplikation zu speichern.

\section{LD10 - Ausgelesene Motordaten aus OBD-II}
Motordaten sind unbedingte Informationen, weshalb diese Funktion ohne einer Verbindung zur OBD-II Schnittstelle nicht lauffähig sein wird.
Sicher erhaltene Informationen aus den Motordaten:
\begin{itemize}
	\item Drehzahl des Motors (Mode 01-PID 0C)
	\item Geschindigkeit des Fahrzeugs (Mode 01-PID 0D)
	\item Kühlwasser Temperatur des Motors (Mode 01-PID 05)
	\item Ansaugklappenstellung (Mode 01-PID 11)
	\item Eingangskraftstoff in \% (Mode 01-PID 2F)
	\item Error Messages/Fehlercodes der OBD-II Schnittstelle (Mode 02)
\end{itemize}
\newpage
Problematischerweise können genau die Informationen über die Schaltvorgänge selbst nicht aus der OBD-II Schnittstelle ausgelesen werden, weshalb ein Schaltvorgang einzig mittels der Ansaugklappenstellung (Throttle Position) und aufgrund der Drehzahl erkannt werden kann.
Die Verbrauchsanalyse ist außerdem bezüglich des verbrauchten Kraftstoffs wahrscheinlich nur mittels dem Wert des Fuel Input zu errechnen, es gibt aber keine vorgefertigten Liter/100km Werte.

\section{LD20 - ausgelesene Rohdaten aus den Sensoren im CarPC}
Rohdaten sind Daten welche noch nicht interpretiert wurden. Diese Rohdaten sind meist Volt-Werte. Sie werden in einem File gespeichert, von dem aus sie interpretiert und in eine Datenbank gespeichert werden.

\section{LD30 - Interpretierte Informationen}
Informationen, welche nach der Filterung, Normierung, Skalierung und Zuordnung strukturiert in der Datenbank abgelegt werden.

\newpage
\section{LD40 - Konfigurationsdaten}
Als Konfigurationsdaten, werden die Daten bezeichnet, welche unvermeindlich vom Benutzer angegeben werden müssen, um Informationen über das Fahrzeug zu erhalten.   Diese Daten sollen strukturiert in der NOSQL Datenbank abgelegt werden, wobei diese entweder aus den Informationen des Motormanagements oder direkt aus dem Input des Users generiert werden.
\nextline
\textbf{Beispiele für Eingaben durch den Benutzer}
\begin{itemize}
	\item Ganganzahl des Getriebes [4-7] (Vorraussetzung für Schaltvorhersage)
	\item \textit{Kraftstoffart [Super/Diesel]}
	\item Beschleunigung 0-100 km/h [s] (Voraussetzung für die Drehmomentoptimierung)
	\item \textit{Durchschnittsverbrauch innerorts [1.0-50.0]} 
	\item \textit{Durchschnittsverbrauch außerorts [1.0-50.0]}	
	\item Durchschnittsverbrauch kombiniert [1.0-50.0] (Voraussetzung für die Verbrauchsoptimierung)
	\item \textit{Hubraum [0.1-12.0]}
	\item \textit{Momentangewicht [300-20000]}
	\item \textit{zulässiges Gesamtgewicht [300-20000]}
	\item \textit{Passagiere [1-70]}
\end{itemize}
\textit{optionale Informationen}

\section{LD50- Verlaufsinformation, Sensordaten über Zeit}
Da die Datenmenge von vielen Fahrten auf dem Rasberry Pi2 schnell zu einer unüberschaubaren Menge führen würde, werden die weiter zurück liegenden Informationen und Daten nach dem Prinzip einer Round-Robin-Database in einer verringerten Auflösung bereitgestellt. 
Aufgrunddessen wird der Nutzer die Möglichkeit haben einzelne interessante Fahrten und Streckendaten zu speichern bzw. zu favorisieren, welche nicht in ihrer Auflösung verringert werden, da dies für alle nicht-markierten Fahrten automatisch passieren wird. Diese favorisierten Streckendaten können dann auch exportiert und auf der Webapplikation oder auch mit anderen Daten(Bildern) analysiert werden.
