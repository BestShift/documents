\chapter{Einleitung}
BestShift ist ein Projekt das dem Fahrer als oberstes Ziel mehr Informationen während- und nach der Fahrt bieten soll. Deshalb werden in unterschiedlichen Teilen des Projektes bis April 2016 (finale Abgabe der Diplomarbeit) diverse einzelne Komponenten einer hybriden Android Applikation und einer Hardware Schnittstelle für das ODB-II Interface entwickelt. 

Für die Applikation wird ein Android Framework verwendet werden, während die Hardware aus einem Single Board Computer (SBC) bestehen soll, welcher mittels einem ODB-II zu Bluetooth Chip (ELM237) und Python, Daten aus dem Motormanagement sowie eigene Sensordaten erhält. 

\newpage
\section{Projekt-Team}
\textbf{Tobias Perny} hat als einziger Systemtechniker der Projektgruppe die hardwarenäheste Aufgabe des ganzen Projektes. Er soll einen CarPC mit allen nötigen Sensoren, aber allen voran einem GPS-Sensor, Beschleunigungssensor und Gyroskop verwirklichen. Dafür muss der CarPC an das 12V Bordnetz eines PKW angebunden werden, außerdem sollen dessen Daten ausgelesen werden.
\nextline 
\textit{Fähigkeiten}
\begin{itemize}
	\item Mikrokontroller
	\item Ansteuerung von Sensoren		
	\item Softwareengineering
	\item Microsoft Office
	\item SCRUM
	\item Informationsmanagement
\end{itemize}

\nextline

\textbf{Daniel Melichar} ist der perfekte Vertreter des Produktes für den Kunden, und bei Beendigung des Projektes werden mögliche Interessenten die über diverse Networks auf das Produkt aufmerksam geworden sind und es gerne kaufen würden mit ihm Kontakt aufnehmen. Herr Melichar ist höchst engagiert für das Projekt und besitzt die Fähigkeiten dieses Engagement in die Tat um zu setzten. Herr Melichar ist im Projekt vor allem für die Bereitstellung und Interpretierung der Motor- und Sensordaten zuständig. 

\textit{Fähigkeiten}
\begin{itemize}
	\item Servermanagement & Datenmanagement				
	\item Softwareengineering
	\item Webentwicklung & -design
	\item SCRUM
	\item Quality Assurance
	\item MicrosoftOffice
\end{itemize}

\newpage

\textbf{Raphael Simsek} hatte gemeinsam mit Herrn Prof. List die Idee für das Projekt entwicklet. Das Teilprojekt von Herrn Simsek umfasst die Berechnung von Schaltvorschlägen mittels Algorithmen für maximales Drehmoment und für kraftstoffsparendes Fahren. Das Wissen über die Algorithmen für diese Berechnungen wird erst erlangt. Das Teilprojekt wird allerdings definitiv für Android programmiert sein und die Drehzahl ebenfalls mittels der Motordaten darstellen.

\textit{Fähigkeiten}
\begin{itemize}
	\item Servermanagement				
	\item Softwareengineering
	\item Webentwicklung
	\item Microsoft Office
	\item Adobe Creative Suite
	\item Urheberrechtliches Consulting
\end{itemize}

\nextline

\textbf{Hüseyin Bozkurt} ist für die Verbrauchsanalyse der App zuständig. Dabei wird eine Anzeige des Momentanverbrauchs und des Durchschnittsverbrauchs der ganzen Fahrt angezeigt. Diese soll sowohl in Werten mittels Graphiken dargestellt werden. Die Fahrstrecke soll auch auf den Verbrauch bezogen geographisch angezeigt werden. Ein Verbrauchsvergleich und eine Social Media Integration ist auch integriert.
All diese geplanten Features wird Herr Bozkurt umsetzten. Die hybride Applikation ist besonders für diese Teilbereiche wichtig, da hier eine Webapplikation, um dem User nach einer Fahrt die Möglichkeit zu bieten, die aufgenommenen Daten und Graphen genauer zu analysieren, realisert ist.

\textit{Fähigkeiten}
\begin{itemize}
	\item Projektmanagement			
	\item Softwareengineering
	\item SCRUM
	\item Microsoft Office
	\item Adobe Creative Suite
\end{itemize}

\newpage

\textbf{Fitim Faiku} ist für die Analyse der Fahrgastbequemlichkeit zuständig. DDieses Feature wird mithilfe eines Kamm'schen Kreises umgesetzt, welcher in der App abgerufen werden kann. Dieser zeigt Quer- und Längsbeschleunigung an. Der Kreis soll je nach Modus die g-Kraft innerhalb von unterschiedlichen Bereichen darstellen. 

\textit{Fähigkeiten}
\begin{itemize}
	\item Projektmanagement			
	\item Softwareengineering
	\item SCRUM
	\item Microsoft Office
	\item Adobe Creative Suite
\end{itemize}

\section{Projektbeschreibung}
In diesem Projekt wird eine Applikation für das mobile Betriebssystem Android implementiert, die dem Fahrer eines KFZ während der Fahrt Informationen zu Fahrgastbequemlichkeit und verbrauchseffizienter Fahrweise gibt. Dazu werden Daten aus dem Motormanagement verwendet. Für weitere benötigte Daten (z.B. Beschleunigungswerte) werden zusätzliche externe Sensoren in einem portablen CarPC integriert. Die App wird zusätzlich zur Anzeige von Momentanwerten während der Fahrt, die Möglichkeit bieten, die gesammelten Daten einer Fahrtstrecke für spätere Analysen graphisch einfach aufbereitet anzuzeigen. Die einzelnen Messwerte sollen dabei mit geographischen Informationen verknüpft werden, um dem Fahrer zum Beispiel zu zeigen, welche Stellen der Strecke besonders verbrauchsintensiv oder unbequem für den Fahrgast waren. 

Als weitere Funktion soll die Applikation aus den ermittelten Daten dem Fahrer den momentan am energieeffizientesten oder leistungsstärksten Gang vorschlagen können.
