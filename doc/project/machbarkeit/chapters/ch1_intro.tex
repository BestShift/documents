\chapter{Introduction}


<<<<<<< HEAD
\section{Projekt-Team}
\textbf{Tobias Perny}
Tobias Perny hat als einziger Systemtechniker der Projektgruppe hardwarenäheste Aufgabe des ganzen Projektes. Er soll einen CarPC mit allen nötigen Sensoren, aber allen voran einem GPS-Sensor, Beschleunigungssensor und Gyroskop verwirklichen. Dafür muss der CarPC an das 12V Bordnetz eines PKW angebunden werden, außerdem sollen dessen Daten ausgelesen werden. 
\textit{Fähigkeiten}
\begin{itemize}
	\item Mikrokontroller
	\item Ansteuerung von Sensoren		
	\item Softwareengineering
	\item Microsoft Office
	\item SCRUM
	\item Data Mining
	\item \todo{Mehr Fähigkeiten für Tobi einfügen}
\end{itemize}

\textbf{Daniel Melichar - Product Owner}

Daniel Melichar ist der perfekte Vertreter des Produktes für den Kunden, welcher momentan unser SCRUM Master und Stakeholder Prof. Erhard List ist und bei Beendigung des Projektes werden mögliche Interessenten die über Mundpropaganda auf das Produkt aufmerksam geworden sind und es gerne kaufen würden mit ihm Kontakt aufnehmen. Herr Melichar ist höchst engagiert für das Projekt und besitzt die Fähigkeiten dieses Engagement in die Tat um zu setzten.

\textit{Fähigkeiten}
\begin{itemize}
	\item Projektmanagement
	\item Servermanagement & Datenmanagement				
	\item Softwareengineering
	\item Webentwicklung
	\item Design
	\item SCRUM
	\item Microsoft Office
	\item Adobe Creative Suite
	\item Quality Assurance
\end{itemize}
Herr Melichar ist im Projekt vor allem für die Bereitstellung und Interpretierung der Motor- und Sensordaten zuständig. 

\newpage

\textbf{Raphael Simsek}
Raphael Simsek hatte gemeinsam mit Herrn Prof. List die Idee für das Projekt und ist grundsätzlich sehr interessiert an Autos, genauso wie Herr Faiku allerdings vorrangig an japanischen Ikonen. Dazu zählen beispielsweise der Toyota Supra, Mazda RX-7 FD, Honda S2000 oder Nissan Skyline R32. Das Teilprojekt von Herrn Simsek umfasst die Berechnung von Schaltvorschlägen mittels Algorithmen für maximales Drehmoment und für kraftstoffsparendes Fahren. Das Wissen über die Algorithmen für diese Berechnungen wird erst erlangt. Das Teilprojekt wird allerdings definitiv für Android programmiert sein und die Drehzahl ebenfalls mittels der Motordaten darstellen.

\textit{Fähigkeiten}
\begin{itemize}
	\item Projektmanagement
	\item Servermanagement				
	\item Softwareengineering
	\item Webentwicklung
	\item Microsoft Office
	\item Adobe Creative Suite
	\item Urheberrechtliches Consulting
\end{itemize}

\textbf{Hüseyin Bozkurt}
Hüseyin Bozkurt ist für die Verbrauchsanalyse der App zuständig. Dabei wird eine Anzeige des Momentanverbrauchs und des Durchschnittsverbrauchs der ganzen Fahrt angezeigt. Diese soll sowohl in Werten als auch in Graphiken dargestellt werden. Die Fahrstrecke soll auch auf den Verbrauch bezogen geographisch angezeigt werden. Beispielsweise mittels eines Overlays. Ein Verbrauchsvergleich und eine Social Media Integration ist auch geplant.
All diese geplanten Features wird Bozzy durch sein erlerntes Wissen am TGM und sein Interesse für die Programmierung mit Bravur umsetzen können. Die hybride Applikation ist besonders für diese Teilbereiche wichtig, da hier eine Webapplikation, um dem User nach einer Fahrt die Möglichkeit zu bieten die aufgenommenen Daten und Graphen genauer zu analysieren, geplant ist.

\textit{Fähigkeiten}
\begin{itemize}
	\item Projektmanagement
	\item Servermanagement				
	\item Softwareengineering
	\item Webentwicklung
	\item SCRUM
	\item Microsoft Office
	\item Adobe Creative Suite
	\item Quality Assurance
	\todo{Mehr Fähigkeiten für Bozzy hinzufügen}
\end{itemize}

\textbf{Fitim Faiku}
Fitim Faiku ist für die Analyse der Fahrgastbequemlichkeit zuständig. Die Umsetzung dieses Features ist vor allem mit einem Kamm'schen Kreis geplant, welcher in der App verwendet werden soll um besondere Quer- oder Längsbeschleunigungen anzuzeigen. Der Kreis soll je nach Modus die g-Kraft innerhalb von unterschiedlichen Bereichen darstellen. 
Herr Faiku, welcher sich selbst sehr für japanische Sportwagen interessiert, insbesondere der Nissan GTR bereitet ihm Freude, ist sehr von Autos begeistert und hat auch die Programmierkenntnisse um die Wünsche des Kunden in die Tat umzusetzen. 

\textit{Fähigkeiten}
\begin{itemize}
	\item Projektmanagement
	\item Servermanagement				
	\item Softwareengineering
	\item Webentwicklung
	\item Microsoft Office
	\item Adobe Creative Suite
	\todo{Mehr Fähigkeiten für Fitim hinzufügen}
\end{itemize}
=======

\section{Projekt-Team}
>>>>>>> parent of 9a52fa7... Added introduction and project team descriptions

\section{Projektbeschreibung}
In diesem Projekt wird eine Applikation für das mobile Betriebssystem Android implementiert, die dem Fahrer eines KFZ während der Fahrt Informationen zu Fahrgastbequemlichkeit und verbrauchseffizienter Fahrweise gibt. Dazu werden Daten aus dem Motormanagement verwendet. Für weitere benötigte Daten (z.B. Beschleunigungswerte) werden zusätzliche externe Sensoren in einem portablen CarPC integriert. Die App wird zusätzlich zur Anzeige von Momentanwerten während der Fahrt, die Möglichkeit bieten, die gesammelten Daten einer Fahrtstrecke für spätere Analysen graphisch einfach aufbereitet anzuzeigen. Die einzelnen Messwerte sollen dabei mit geographischen Informationen verknüpft werden, um dem Fahrer zum Beispiel zu zeigen, welche Stellen der Strecke besonders verbrauchsintensiv oder unbequem für den Fahrgast waren. 

Als weitere Funktion soll die Applikation aus den ermittelten Daten dem Fahrer den momentan am energieeffizientesten oder leistungsstärksten Gang vorschlagen können.
