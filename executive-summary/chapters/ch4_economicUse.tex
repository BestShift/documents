\chapter{wirtschaftlicher Nutzen}

\section{Vermarktung}
Laut Business Plan soll unser äußerst innovatives Produkt hauptsächlich mittels unserer Social Media Anbindung Verbreitung finden. Es finden ausserdem bereits Überlegungen statt das fertige Produkt bekannten Persönlichkeiten, z.B. Bloggern, kostenlos zur Verfügung zu stellen, um damit neue Kunden anzuziehen.
\newline
Bei großen Firmen planen wir die Vermarktung des Produktes auf einer eher persönlichen Ebene indem wir mit den jeweiligen Firmenverantwortlichen Treffen organisieren, bei welchen wir durch Briefing und Debriefs der vorherigen Meetings immer perfekt vorbereitet sein wollen.
\section{Anwendungsgebiete}
\subsection{Endverbraucher}
Für Endverbraucher planen wir das Produkt als Einzellizenz verfügbar zu machen. Dabei erhält der Kunde bei längerfristigen (i.d.R. 1 Jahr) Abonnements Vergünstigungen. Ausserdem ist eine Vergünstigung für Schüler und Studenten geplant, da das Produkt ja auch Fahranfängern helfen soll und das Feedback von \textit{digital natives} uns besonders bei der weiteren Verbesserung des Produktes helfen könnte.
\subsection{Firmenkunden}
Für Firmenkunden haben wir, wie bereits erwähnt, geplant, diese mittels persönlichen Gesprächen von der Innovationsstärke unseres Produktes und der Nutzen aus unserem Produkt überzeugen zu können. 
Im speziellen stechen für uns Fahrschulen und Logistik Unternehmen als mögliche Zielgruppe heraus. 
\newline
Fahrschulen können das Produkt verwenden um Ihren Fahrschülern  mittels der Analyse Funktionalität einen leichteren Einstieg in das Autofahren bieten zu können, welches diese als Alleinstellungsmerkmal vermarkten können.
\newline
Speditionen können das Produkt verwenden um Spritkosten einzusparen indem die Fahrer einer Spedition dazu angehalten werden möglichst spritsparend zu fahren, wobei Ihnen die Applikation und dessen Analysefunktionalität helfen soll. Wenn gewünscht kann für den ökologischten Fahrer beispielsweise auch eine firmeninterne Prämie ausgegeben werden.