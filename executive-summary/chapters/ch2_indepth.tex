\chapter{Einführung}
\section{Analyse und Verbesserung des Fahrstils}
Oft können Fahrlehrer nicht einen perfekten Einstieg in die komplexe Praxis von effizientem Fahrstil geben. Der Fahrer tut sich dann nach langer Gewöhnung an seinen Fahrstil, schwer diesen kurzfristig zu ändern. Daher soll unser Produkt dem Fahrer die Chance geben längerfristig an seinem Fahrstil zu arbeiten und seine eigenen Fehler nachhaltig erkennen zu können.

\newline
Diese Einbindung des Fahrstils des Fahrers wird bei BestShift mit der Webapplikation umgesetzt. Ausserdem sieht der Fahrer in Echtzeit mögliches Verbesserungspotenzial über Graphen der Bequemlichkeit und des Schadstoffausstoßes. Zusätzlich wird Fahranfängern ein Schaltvorschlag geboten, um noch bei der Persistierung des Fahrstils Einfluss im Bezug auf den Wirkungsgrad des Motors nehmen zu können. 