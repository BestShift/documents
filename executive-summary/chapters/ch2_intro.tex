\chapter{Einführung}

\section{Analyse und Verbesserung des Fahrstils}
Oft können Fahrlehrer nicht einen perfekten Einstieg in die komplexe Praxis von effizientem Fahrstil geben. Der Fahrer tut sich dann nach langer Gewöhnung an seinen Fahrstil, schwer diesen kurzfristig zu ändern. Daher soll unser Produkt dem Fahrer die Chance geben längerfristig an seinem Fahrstil zu arbeiten und seine eigenen Fehler nachhaltig erkennen zu können.

\newline
Diese Einbindung des Fahrstils des Fahrers wird bei BestShift mit der Webapplikation umgesetzt. Ausserdem sieht der Fahrer in Echtzeit mögliches Verbesserungspotenzial über Graphen der Bequemlichkeit und des Schadstoffausstoßes. Zusätzlich wird Fahranfängern ein Schaltvorschlag geboten, um noch bei der Festigung des Fahrstils Einfluss im Bezug auf den Wirkungsgrad des Motors nehmen zu können.

\newline
\section{Umweltbelastung}
Der CO2 Ausstoß ist besonders bei Festigung eines eher aggressiven Fahrstils oft sehr hoch. Dagegen schafft BestShift Abhilfe, indem dem Nutzer der momentane Kraftstoffverbrauch angezeigt wird und ihm ein Rating gegeben wird, dass er auch teilen kann. So wollen wir möglichst viele Personen davon überzeugen ökologisch zu fahren. 

\section{Das Projekt BestShift}
\subsection{Zielsetzung}
Das Diplomprojekt hat sich als oberstes Ziel gesetzt, dass ein Computer entwickelt wird, der in einen DIN Slot passt, eine funktionierende Android App programmiert wird und dass die Kommunikation zwischen Dateninterface und Applikation problemlos funktioniert.
\subsection{Erreichte Ziele}
Erreicht wurde ein fertiger Car PC, der inkl. Beschleunigungs- und Temperatursensor, welcher ausserdem mit einem Akku, einem Strom Pi und WiFi und Bluetooth ausgestattet wurde und dadurch alle Möglichkeiten zur Kommunikation zwischen Applikation und dem eigenen Dateninterface bietet.
Ausserdem wurde die Android App abgeschlossen und kann nach kleinen Änderungen die Kommunikation mit dem Dateninterface aufbauen.
Die Kommunikation muss noch für eine aus dem Prototypenstadium ausgebaut werden. Die Datenbank und Datenverbindung ist realisiert worden.


\section{Projektmanagement}
Wir haben uns entschieden für das Projekt agiles Projektmanagement zu verwenden, wobei das im Unterricht kennengelernte SCRUM für uns besonders passend erschien. Gründe dafür waren under anderem dass wir nicht bis ins Detail wussten wie viel Arbeit auf uns zu kam und die Materie mit der wir uns befassten für uns sehr neu war.
\newline
Die Arbeitspakete (\textit{User Stories}) wurden in 5 Sprints á 3 Wochen eingeteilt. Für uns erschienen 3 Wochen bei der Kürze des Projektes die einzige passable Option zu sein.
Als Artefakte wurden ein Lastenheft, eine Machbarkeitsstudie, Sprintabnahmen, Retrospektiven und einige Benutzerhandbücher erstellt.