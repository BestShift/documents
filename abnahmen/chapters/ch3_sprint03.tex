\chapter{Sprint 03: 13. Jan, 2016}

\textbf{Sprint beginn:} 13.01.2016
\nextline
\textbf{Sprint ende:} 19.03.2016

\begin{tabular}{|l|l|l|l|l|l|}
            & Datumn        & Name            & \multicolumn{3}{l}{Unterschrift} \\ \hline
Erstellt    &  13.Jan.2016  & Faiku Fitim     & \multicolumn{3}{l}{}             \\ \hline
Geprüft     & 13. Jan, 2016 &                 & \multicolumn{3}{l}{}             \\ \hline
Freigegeben &               &                 & \multicolumn{3}{l}{}            
\end{tabular}

\subsection{Burndownchart}
%\includegraphics[scale=0.2, ]{images/sprint03-burndown.png}

\subsection{User Stories}
\resizebox{\columnwidth}{!}{%
\begin{tabular}{|l|l|l|l|l|l|}
\hline
\textbf{\#} & \textbf{User Story}                                                                                                                                                                                                                                                                                                                                                                                                                         & \textbf{Story points} & \textbf{Verantwortung} & \textbf{Akzeptanz} & \textbf{Kommentar} \\ \hline
\hline
27 & \begin{tabular}[c]{@{}l@{}}Car-PC Manual\\ \\ Als User möchte ich eine Anleitung zum Bedienen des Car Pc im Lieferumfang enthalten haben weil ich nicht weiß wie ich den Car PC im KFZ installieren kann/soll\\ \\ Akzeptanzkriterien:\\ - Abdeckung der Autos\\ - Detailierungsgrad anwendbar auf echtes Auto\end{tabular} & 2 & Tobi &  &  \\ \hline
3 & \begin{tabular}[c]{@{}l@{}}Audiovisuelles-Feedback: Beschleunigungskraft\\ \\ Als User möchte ich eine aktuelle audiovisuelle Anzeige aller Beschleunigungskräfte haben, da ich somit aktuell darauf reagieren kann\\ \\ Akzeptanzkriterien:\\ - Ton wird als negatives Feedback erkannt.\end{tabular} & 3 & Fitim &  &  \\ \hline
20 &\begin{tabular}[c]{@{}l@{}} Kammscherkreis\\
Als User möchte ich für eine bestimmte Fahrt einen Kammschen Kreis betrachten können, da ich somit die Bequemlichkeit meiner Fahrten analysieren kann.
 \\ \\ Akzeptanzkriterien: \\ - Responsiv Design \\
 -beschriftet(Titel, x-y Achse) .\end{tabular} & 8 & Fitim &  &  \\ \hline
29 & \begin{tabular}[c]{@{}l@{}}Sensordaten aufbereitet\\ \\ Als User möchte ich die von den Sensoren erhaltenen Rohdaten aufbereitet und gefiltert dargestellt bekommen.\\ \\ AKZEPTANZKRITERIEN\\ -  Sensordaten aus Zwischen-DB lesbar\\ 
	- Sensordaten auf das Notwendigste beschränkt \end{tabular} & 8 & Daniel &  &  \\ \hline

183 & \begin{tabular}[c]{@{}l@{}}PC in DIN Slot Größe\\ \\ Als Entwickler möchte ich einen Car-PCs entwerfen, welcher Platz in einem KFZ DIN Slot hat. \\ \\
	Akzeptanzkriterien: \\
	-Schematik vorhanden\\
	-Car-PC Fertiggestellt \end{tabular} & 8 & Tobi &  &  \\ \hline
15 & \begin{tabular}[c]{@{}l@{}}Datenspeicherung der Sensoren\\ \\ Als Entwickler möchte ich eine Datenbank zum handling der Sensordaten erstellen weil\\ fast jedes Features die aufbereiteten Daten benötigt.\\ \\ Akzeptanzkriterien: \\ - Datenschema auf andere DBMS übertragbar\\ - Neue Sensoren können in die DB speichern\\ - Sensordaten wurden gefiltert\end{tabular} & 8 & Daniel &  &  \\ \hline
6 & \begin{tabular}[c]{@{}l@{}}Sicherung der Sensordaten\\ \\ Als Admin möchte ich einen Datenbank dump \\erstellen können weil ich die Daten damit\\ zu einem späteren Zeitpunkt wieder einfügen kann, bzw. auf einen anderen migrieren.\\ \\ Akzeptanzkriterien: \\ - Sicherung von spezifischen bzw. allen Daten\\ - Sensordaten verschlüsselt\\ - Sensordaten einfach auf (anderes) System migrierbar\end{tabular} & 3 & Daniel &  &  \\ \hline
96 & \begin{tabular}[c]{@{}l@{}}User Datenbank\\ \\ Als Entwickler möchte ich eine Datenbank besitzen die zur Verwaltung der User dient.\\ \\ Akzeptanzkriterien:\\ - Einheitliches Datenschema\\ - Dateneinträge modifizierbar \\ -Dateneinträge Löschbar \end{tabular} & 8 & Daniel &  &  \\ \hline
8 & \begin{tabular}[c]{@{}l@{}}Kammscher Kreis (Handy)\\ \\ Als User möchte ich mittels eines Buttons einen Kammschen Kreis am Ende der aktuellen\\ Fahrt betrachten können, da ich\\ somit die Beschleunigungskräfte meiner Fahrt betrachten kann\\ \\ Akzeptanzkriterien:\\ Wohlfühlbereich angezeigt
	\end{tabular} & 8 & Fitim &  &  \\ \hline
9 & \begin{tabular}[c]{@{}l@{}}Einfaches Einbauen des Car-PCs\\ \\ Als User möchte ich den Car PC ohne großen Aufwand im KFZ installieren können,\\ weil ich kein Gerät benötige, welches mir unnötige Arbeit bringt. \\ \\
	Akzeptanzkriterien: \\
	Fahrzeugunabhängig einbaubar\\
	Alternativen möglich \end{tabular} & 3 & Tobi &  &  \\ \hline


17 & \begin{tabular}[c]{@{}l@{}}Feedback: Temperatur \\ \\ Als User möchte ich aktuelles Feedback über die Temperatur der Fahrerkabine bekommen, \\ da ich dadurch weiß wann die Temperatur bequem ist oder nicht.\\ \\ AKZEPTANZKRITERIEN:\\ - Verschiedene Farben beim anzeigen verwendet. \end{tabular} & 5 & Fitim &  &  \\ \hline

7 & \begin{tabular}[c]{@{}l@{}}Gangvorschlag:Hochschalten \\ \\ Als User möchte ich sofort \\ durch das Geräusch der App erkennen ob ich hoch schalten sollte weil ich während der Fahrt nicht abgelenkt werden möchte. \\ \\ AKZEPTANZKRITERIEN:\\ - einzigartiger Sound innerhalb der App. \end{tabular} & 8 & Raphael &  &  \\ \hline

5 & \begin{tabular}[c]{@{}l@{}}Sensordaten export \\ \\ Als User möchte ich meine Fahrdaten aus der Handyapp exportieren können weil ich sie auf der Webapp visualisieren möchte. \\ \\ AKZEPTANZKRITERIEN:\\ - Lokale Speicherung bei keiner Internet Verbindung \\ -File Größe des Exports <= 50MB \end{tabular} & 13 & Daniel &  &  \\ \hline

\end{tabular}
}