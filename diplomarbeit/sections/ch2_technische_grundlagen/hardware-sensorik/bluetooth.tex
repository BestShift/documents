\lfoot{Autor: Tobias Perny}
\subsubsection{Bluetooth}
\label{subsec:Bluetooth}
Bluetooth ist eine kabellose Variante wie mehrere Geräte über kurze Distanz miteinander Kommunizieren können.

\begin{wrapfigure}{r}{0.2\textwidth}
  \begin{center}
    \includegraphics[width=0.2\textwidth]{images/bluetooth}
  \end{center}
  \caption{Bluetooth Logo, inspiriert vom König von Norwegen und Dänemark, Harald Blauzahn \cite{PERT.CH2-bluetooth.logo}}\label{Fig:imgBluetoothLogo}
\end{wrapfigure}

\textbf{Kommunikation\newline}
Eine Kommunikation mittels Bluetooth kann man in folgende Schritte unterteilen.
\nextline
\begin{itemize}
\item Ein Gerät auswählen mit welchem kommuniziert werden soll
\item Herausfinden, wie man mit dem besagten Gerät kommunizieren kann
\item Aufbauen einer ausgehenden Verbindung
\item Akzeptieren einer ankommenden Verbindung
\item Senden und empfangen von Daten
\end{itemize}
\textbf{Gerätewahl\nextline}
Jeder Bluetooth Chip besitzt eine eindeutige Adresse. Damit ein Gerät mit einem anderen Kommmunizieren kann, muss es eine Mögllichkeit geben, diese Bluetooth-Adresse zu erfahren. Im Fall von Bluetooth werden dem Gerät meist benutzerfreundliche Namen vom Besitzer oder der Besitzerin vergeben.
Wahl des Transportprotokolles
Die Wahl des Transportprotokolles ist entscheidend für die Art der Datenübertragung. Es gibt mehrere Transportprotokolle, hier werden zwei exemplarisch aufgeführt.
\nextline
\textbf{RFCOMM\nextline}
Bei RFCOMM wird Wert gelegt auf übertragungs Zuverlässigkeit. Es sollen möglichst wenig bis gar keine Pakete verloren gehen. Wenn eine gewisse Menge an Daten nicht in einer bestimmten Zeit gesendet werden kann, wird die Verbindung getrennt und muss neu aufgebaut werden.
\nextline
\textbf{L2CAP\nextline}
Bei diesem Protokoll kommt es nicht auf den Datenverlust an sondern auf die Tatsache das die verbindung stehts aufrecht erhalten wird. Es wird ein Paket so oft gesendet bis es ankommt oder die Verbindung komplett ausfällt.