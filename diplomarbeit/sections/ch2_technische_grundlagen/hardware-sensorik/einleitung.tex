\lfoot{Autor: Tobias Perny}
\subsubsection{Einleitung}
\label{subsec:Einleitung}

Da das Projekt Bestshift auch die Darstellung von Beschleunigungsdaten, sowie der Innenraumtemperatur umfasst, ist es notwendig eigene Hardware in Form von Sensoren zum KFZ hinzuzufügen.
Hierfür standen mehrere Varianten zur Auswahl:

\begin{itemize}
	\item Smartphone:
	\begin{itemize}
	\item Vorteile: \nextline
	Grundvoraussetzung für den Anwender unseres Systems ist der Besitz eines Smartphones mit einem Android Betriebssystem.
	Das bedeutet, in jedem KFZ unserer Anwender befindet sich ein Smartphone aus welchem Sensorenwerte ausgelesen werden könnten.
	\item Nachteile: \nextline
	Die Sensoren müssen an einer Position im KFZ still stehen und dürfen sich während der Fahrt nicht bewegen. Dies würde bedeuten, dass das Smartphone die gesamte Fahrt über, nicht verwendet werden kann.
\nextline
Desweiteren ist die Auflösung der Sensoren in verschiedenen Smartphones nicht ausreichend für den Anwendungsfall dieses Projektes.
	\end{itemize}
	\item Selbstgebauter Car-PC:
	\begin{itemize}
	\item Vorteile: \nextline
	Einen eigenen PC zu bauen ermöglicht uns die Wahl der passenden Sensoren. Das bedeutet, in jedem Fahrzeug befinden sich der gleiche Car-PC, also dieselben Sensoren.
	Zusätzlich dazu ist es nun bei dieser Variante auch möglich, eine Datenbank auf dem Car-PC zu installieren, und somit Speicher auf dem Smartphone des Benutzers zu sparen.
	\item Nachteile: \nextline
	Ein proprietärer Car-PC muss selbst im Fahrzeug eingebaut werden. Um diesen Nachteil auszugleichen, ist der PC so konzipiert, dass er möglichst einfach in einem DIN-Slot installiert werden kann.
	\end{itemize}
\end{itemize}