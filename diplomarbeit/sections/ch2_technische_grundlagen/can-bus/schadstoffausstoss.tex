\lfoot{Autor: Faiku Fitim}
\subsubsection{Schadstoffausstoss}
\label{subsec:schadstoffausstoss}

Bei einem Verbrennungsmotor wird ein Kraftstoff-Luft-Gemisch in einem Zylinder gezündet und verbrannt.
Alle Verbrennungsmotoren wiederholen dies in einem Kreisprozess, ständig im Arbeitszyklus, der aus vier Arbeitsschritten besteht.

\textbf{1. Ansaugen\newline}
Ausstoßen und Ansaugen dienen dem Gangwechsel, das heißt, dem Austausch von Abgas(Ausstoßen) gegen Frischgas (Ansaugen). Hierfür ist die Motorsteuerung zuständig.
\textbf{2. Verdichten\newline}
Verdichten und Arbeiten dienen der Umwandlung von chemischer Energie(Verbrennung des Gemisches) über thermische Energie (Wärme) und potentielle Energie(Druck) in mechanische Energie (Drehmoment).

\textbf{3. Arbeiten\newline}
Das wird oft als Takt bezeichnet.

\textbf{4. Ausstoßen \newline}
Der sich nach oben bewegende Kolben schiebt im vierten Takt die verbrannten Abgase aus dem Zylinder durch den Auspuff in die Umwelt. 
Die Abgastemperatur liegt bei 800 Grad im Leerlauf, bis 1000 Grad bei Vollast. 
\newline 



\clearpage % DO NOT REMOVE