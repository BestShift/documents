
\lfoot{Autor: Fitim Faiku}
\subsubsection{Grundsätze einer Android Applikation}
\label{subsec:aapp-fundam}

Android Apps werden in der Programmiersprache Java geschrieben.
Die Android SDK-Tools kompilieren Code zusammen mit allen Ressourcen-Dateien in eine APK.
Eine APK ist ein Android-Paket, welches alle Inhalte einer Android App enthält. 
Die APK-Datei wird dann auf das Android-Gerät installiert. 


\textbf{Native vs HTML5 vs Hybrid}
Bei der Umsetzung unserer Android Applikation, hatten wir die Entscheidung, wie wir die Android App erstellen. 
Da Android mehrere Programmierarten unterstützt musste Evaluiert werden welche Programmiermodule für unsere App am Besten geeignet ist. \nextline

\textbf{Native\newline} 
Eine Native App wird fuer die IOS und Android Plattformen verwendet wobei die Programmiersprachen dafür Objektiv C und Java sind.
Native Apps Performen am Besten.
Die Dokumentation zu Native Apps ist am Besten, da es über 2500 Bücher zu IOS und Android entwicklung gibt.
Fuer Nativ Apps werden sogennante Entwicklungsumgebungen oder auch genannt IDE verwendet um auf denen die gewünschte App zu Programmieren.\nextline

\textbf{HTML5\newline} 
HTML5 verwendet die Standard web Technologien, wie HTML5, JavaScript und CSS. Das gute an HTML5 Apps ist, dass es auf verschiedenen Plattformen verwendet werden kann.\nextline

\textbf{Hybrid\newline} 
Hybrid Apps sind eine Mischung von Nativ und HTML5 Apps. Hybrid bietet das Beste aus beiden, schön gebaute CSS layout und das schreiben kompatible HTML-Code, der auf jeder Plattform funktioniert.\nextline

\clearpage % DO NOT REMOVE