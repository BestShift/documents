
\lfoot{Autor: Fitim Faiku}
\subsubsection{Grundsätze einer Android Applikation}
\label{subsec:aapp-fundam}

Android Apps werden in der Programmiersprache Java geschrieben.
Die Android SDK-Tools kompilieren Code zusammen mit allen Ressourcen-Dateien in eine APK.
Eine APK ist ein Android-Paket, welches alle Inhalte einer Android App enthält. 
Die APK-Datei wird dann auf das Android-Gerät installiert. 


\textbf{Native vs HTML5 vs Hybrid}
\todo{Überschrift anders?}
Bei der Umsetzung unserer Android Applikation, hatten wir die Entscheidung, wie wir die Android App erstellen. 
Da Android mehrere Programmierarten unterstützt musste evaluiert werden welche Programmiermodule für unsere App am Besten geeignet ist. \nextline

\textbf{Native\newline} 
Für native Apps wird auf der Android-Plattform die Sprache Java verwendet.
Native Apps zeigen die beste Performance.
Die Dokumentation zu Native Apps ist am Besten, da es über 2500 Bücher zur Android Entwicklung gibt.
Für Nativ Apps werden Entwicklungsumgebungen-IDE(Integrated Developement Enviroment) verwendet um die gewünschte App zu Programmieren.\nextline

\textbf{HTML5\newline} 
HTML5 verwendet die Standard Web Technologien, wie JavaScript und CSS.
HTML5 Applikationen werden über Browser abgespielt, wobei jedes Gerät auf die Browser zugreifen kann und dadurch die App auf mehreren Plattformen abgespielt werden kann. \nextline

\textbf{Hybrid\newline} 
Hybrid Apps sind eine Mischung aus nativ und HTML5 Apps. Hybrid bietet das Beste aus beiden Welten. Das Design wird mittels HTML5 und CSS erstellt, wobei sich dahinter ein JavaScript-Code befindet. Hybrid Apps sind genauso auf jede Plattform abspielbar, wobei sie für die Plattform entsprechend umprogrammiert müssten.    
\nextline

\clearpage % DO NOT REMOVE