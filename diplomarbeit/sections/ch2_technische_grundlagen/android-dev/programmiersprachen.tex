\lfoot{Autor: Hüseyin Bozkurt}
\subsubsection{Mitwirkende Programmiersprachen}
\label{subsec:androidsprachen}
\textbf{Entwicklungsumgebung\newline}
Android Studio ist die offizielle Entwicklungsumgebung für Android\texttrademark Applikationen, welche direkt von Google\texttrademark unterstützt wird.
Android Studio basiert auf IntelliJ IDEA und ersetzt das Android Developer Tool für Eclipse, da dieses seit Ende 2015 keinen Support mehr durch Google\texttrademark erhält. Der Funktionsumfang von Android Studio ist:
\begin{itemize}
	\item Build-Management-Automatisierung: Basierend auf Gradle, Entwicklungsmöglichkeiten für verschiedene Gerätetypen (Tablets etc.)
	\item Android Wear/Android TV: Stets auf dem neuesten Stand der Technik können Applikationen für neue Geräte implementiert werden.
	\item Sourcecode-Editor (IntelliJ): Textbearbeitungsoberfläche mit einem Kompiliermechanismus
	\item Smart-Editing: Hilfreiche Vorschläge zur Vervollständigung von Quellcode
	\item Code-Refactoring: Formatiert den Quellcode
	\item Open-Source Entwicklung: Frei verfügbarer Quellcode
	\item GUI-Builder: Einfache graphische Oberflächenentwicklung mittels Drag \& Drop
	\item Zugriff auf Google Dienste: Die Möglichkeit besteht, Google Dienste innerhalb der IDE zu konfigurieren und in der App anzuwenden
	\item ProGuard: Automatische App-Signierung
\end{itemize}

\textbf{Systemvoraussetzungen\newline}
\begin{itemize}
	\item mind. 2GB RAM, 4 GB empfohlen
	\item mind. 400MB freien Speicherplatz (1GB für die Android SDK)
	\item Java Development Kit 7 oder höher
	\item mind. 1280*800 Pixel Bildschirmauflösung
\end{itemize}
\textbf{Programmiersprachen\newline}
Bei den Programmiersprachen bestand die Auswahl zwischen der Programmiersprache Python, 
nativer Javaentwicklung und Hybrid-Java in Verbindung mit HTML5.
Dabei fiel für uns vor allem aufgrund der Verbindung mit einer Webapplikation die Wahl auf eine hybride Applikation. 

\clearpage
\subsubsection{Vergleich Java und Python}
\textbf{Java\newline}
\begin{itemize}
	\item Charakteristik: Mehrfachverwendung/Mehrfachvererbung eingeschränkt, stark typisiert
	\item Syntax: Vor der Kompilierung sehr streng, komplizierte Polymorphie
	\item Dokumentation: Ausführliche Dokumentation der Methoden in Form von Java API
	\item Community: Java ist sehr weit verbreitet, rund 3 Milliarden Geräte können Java ausführen.
	\item Usage: plattformunabhängige Programme, FAT-Client-Anwendungen
	\item Können im Team: 5/5*, jahrelange Erfahrung innerhalb des Teams
\end{itemize}

\textbf{Python\newline}
\begin{itemize}
	\item Charakteristik: Erstellung von Variablen nicht Datentyp abhängig, dynamisch/schwach typisiert.
	\item Syntax: erhöhte Lesbarkeit im Vergleich zu allen anderen, aber ungewohnter zu schreiben als Java
	\item Dokumentation: innerhalb der Python API vorhanden.
	\item Community: Die Community von Python ist ständig am wachsen, ca. 5 Millionen Geräte verwenden Python.
	\item Usage: Back-End, äußerst schnelle Entwicklung (Rapid Application Development)
	\item Können im Team: 3/5*, Team konnte erst geringe Erfahrung mit Python sammeln
\end{itemize}

Aus den oben genannten Punkten filtern wir heraus, dass Java die bessere Lösung für die Android App wäre. 
Die komplette Beherrschung von Python würde einige Wochen dauern. 
Da diese Zeit im Rahmen des Diplomprojektes nicht vorhanden ist, wäre die App-Entwicklung in Python nicht zielführend. 
Außerdem benutzen Android-Smartphones die Dalvik-VM, welche Java Class Files ausführt, wodurch Java nativ auf Android realisiert wird im Gegensatz zu Python. 
Dementsprechend wird hier massiv an Umkompilierungszeit gespart, welche für die Echtzeit-Darstellung des Verbrauchs, Kamm'schen Kreises und Schaltvorschlags essentiell ist.

\clearpage
\subsection{Java-Frameworks (APP)}
\textbf{Frameworks im Allgemeinen\newline}
Ein Softwareentwickler stößt relativ schnell während seiner Ausbildung
auf das Wort \"Framework\". Die Übersetzung ins Deutsche entspricht einem Bilderrahmen oder eines Gerüstes.
Ein Framework ist ein vorgefertigter Baustein, der das Programmieren erleichtern und dadurch Zeit sparen soll.
Vorteile von Frameworks:
\begin{itemize}
	\item Ein Framework besitzt einheitliche Schnittstellen wie zum Beispiel zu Datenbanken, welche die Kommunikation vereinfachen. 
	\item Wiederverwendung von Programmen 
	\item Erleichterung der Programmierarbeit
	\item Einsparung von viel Zeit
\end{itemize}

\begin{table}[!htb]
\centering
\caption{Vergleich der Charting-Frameworks}
\label{comparisonCharting}
\resizebox{\columnwidth}{!}{%
\begin{tabular}{l|c|c|c|l|}
\cline{2-5}
 & \begin{tabular}[c]{@{}c@{}}MPAndorid\\ Chart\end{tabular} & \begin{tabular}[c]{@{}c@{}}AChart\\ Engine\end{tabular} & HelloChart & Erklärung \\ \hline
\multicolumn{1}{|l|}{Performance} & Gut & Gut & Gut &  \\ \hline
\multicolumn{1}{|l|}{\begin{tabular}[c]{@{}l@{}}Supported \\ Languages\end{tabular}} & \begin{tabular}[c]{@{}c@{}}iOS 7,8;\\ Java\end{tabular} & Java & Java & \begin{tabular}[c]{@{}l@{}}MPAC kann auch\\ weiterverwendet werden\\ für eine iOS App\end{tabular} \\ \hline
\multicolumn{1}{|l|}{Documentation} & 85\% & 70\% & 50\% & \begin{tabular}[c]{@{}l@{}}MPAC hat eine sehr gute\\ in Kapiteln aufgelistete\\ Dokumentation, Tutorials\end{tabular} \\ \hline
\multicolumn{1}{|l|}{Recent activity} & 19.10.2015 & 26.07.2013 & 5.10.2015 & \begin{tabular}[c]{@{}l@{}}MPAC ist sehr aktuell\\ und bringt immer wieder \\ neue Bugfixes raus.\\ HelloChart ebenso.\end{tabular} \\ \hline
\multicolumn{1}{|l|}{Community} & 80\% & 60\% & 20\% & \begin{tabular}[c]{@{}l@{}}Bei MPAC und \\ AChartEngine wird eine\\ Antwort innerhalb von 7 \\ Tagen geliefert.\\ Eine Community bei Hello\\ Chart ist kaum vorhanden.\end{tabular} \\ \hline
\multicolumn{1}{|l|}{Usage examples} & 85\% & 70\% & 15\% & \begin{tabular}[c]{@{}l@{}}MPAC hat für jeden Chart\\ ein eigenes Beispiel oder \\ auch Tutorials auf YouTube.\\ AChartEngine weniger \\ verwendbare Codebeispiele.\end{tabular} \\ \hline
\end{tabular}
}
\end{table}

Im Großen und Ganzen ist MPAndroidChart die beste Wahl, da es eine der besten Dokumentationen hat, weiterhin hat es auch eine sehr gute Community, sehr viel gut gecodete und dokumentierte Beispiele und kann  auch erweitert werden, da das Framework auch auf iOS verwendet werden kann. 
AChartEngine hat eine gute Dokumentation, doch eine eher schwache Community und ist leider nicht aktuell. 
HelloChart hat eine der schlechtesten Dokumentationen, gar keine Community und sehr wenig gecodete Beispiele, diese sind aber sehr aktuell.
 


\clearpage % DO NOT REMOVE