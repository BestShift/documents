\subsection{Analyse und Verbesserung des Fahrstils}
Leider kann nicht immer der Fahrlehrer einem Fahrschüler einen nachhaltigen oder wünschenswerten Fahrstil näher bringen kann, da dies oft auch ein andauernder Prozess ist. Zu diesem Zweck wurde sich darüber Gedanken gemacht wie man diese Situation verbessern kann.
\newline
\newline
Vor der Antragstellung wurde überlegt, ob es bereits bekannte Systeme, um nachhaltig seinen Fahrstil zu verbessern, gibt. Es wurde dabei herausgefunden dass es bereits in vielen anderen Bereichen Apps gibt, welche sich das \textit{Sharing}-Prinzip zu nutze gemacht haben und so äußerst bekannt geworden sind. Eine der bekanntesten österreichischen Beispiele ist das, stetig erweiterte, App-System von Runtastic. \cite{Businessplan.Runtastic}
Nur im Bezug auf das Autofahren konnte in diese Richtung nichts lokalisiert werden.
\newline
Zu Beginn des Diplomprojektes gab es also keine Möglichkeiten für verbesserungswillige Autofahrer ihren Fahrstil zu verbessern und dies dann sogar zu teilen. 
Dies hat unsere Marktumfeldanalyse ergeben, wobei uns besonders auffiel, dass eine Analyse-Funktion für die nachhaltige Verbesserung des Fahrstils nötig wäre. Es wurde für diese Analyse evaluiert, welche Funktionalitäten am wichtigsten sind, wobei diese auch innerhalb der Diplomarbeit realisierbar sein sollten.
\newline
\newline
Der bisherige Schaltvorschlag in einem Auto ist, insbesondere bei niedrigpreis Fahrzeugen, oft äußerst primitiv gehalten, wobei schlicht ab einer gewissen Drehzahl die darauf folgende Gänge vorgeschlagen werden. Der Wirkungsgrad nach Carnot-Prozess, wird dabei aber kaum verwendet. \cite{Carnot.Wirkungsgrad}
\newline
\newline
Eine Analyse des Fahrstils, nach Parametern der Fahrgastbequemlichkeit, ist gar nur bei Sportwagen und Fahrzeugen, die für den Rennbetrieb gedacht sind, verbaut. Hierbei ist anzumerken, dass bei derartigen Fahrzeugen die Funktionalität nicht für das frühzeitige Verbessern des Fahrstils eines Fahrers verwendet wird, sondern für die Optimierung von Rundenzeiten auf einer Rennstrecke oder schier für das Ausreizen des Möglichen.
\newline
\newline
Eine Live-Messung von CO2 Werten während der Fahrt ist momentan eine Schätzung, welche durch die Fahrzeughersteller aufgrund der OBD-II Daten durchgeführt und bei sehr wenigen Modellen als Zusatz angezeigt wird. Echte Grenzwerte oder gar eine Analysefunktionalität gibt es in dieser Hinsicht aber nicht.
\newline
\newline
Weil aber besonders Fahranfänger und Fahrer, die große Acht auf die Umwelt legen, eben genannte Analysefunktionalität möglicherweise gerne nutzen würden, wurde ebenfalls evaluiert ob es diese Möglichkeiten bereits gibt.
Bei der Evaluierung wurde ein hardware-seitig sehr ähnliches Projekt entdeckt, welches die gelieferten Daten allerdings nicht auf einer Android\textregistered App verarbeitet, sondern eher als Logger dienen soll.\cite{LowBudget.CarPC} 
\newline
Zusätzlich dazu wäre dieser Car-PC alleine im Bezug auf seine Hardware weitaus teurer als durch dieses Projekt angedacht. Es war aber durchaus möglich in der Definitionsphase des Projektes einige Anpassungen zu treffen, da schon schlussgefolgert werden konnte was im Bereich des möglichen liegt. 
\newline
