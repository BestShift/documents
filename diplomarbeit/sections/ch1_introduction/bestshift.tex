\lfoot{Autor: Faiku Fitim}
\subsection{Das Projekt BestShift}

BestShift ist ein Projekt das dem Fahrer als oberstes Ziel mehr Informationen während- und nach der Fahrt bieten soll. 
Deshalb werden in unterschiedlichen Teilen des Projektes 
diverse einzelne Komponenten einer hybriden Android Applikation und einer Hardware Schnittstelle für das ODB-II 
Interface entwickelt. Der Fokus des Medientechnischen Aspektes dieses Projektes soll allerdings auf der Webapplikation, die das Team ebenfalls realisiert hat. 
Die Web-App ist dafür gedacht, dass Fahrten retrospektiv betrachtet werden können.
Wodurch wir uns besonders hervorheben ist, dass wir das anlegen und verwalten von Accounts ermöglichen welches mit einer Social-Media Anbindung ausgestattet ist.
 

Für die Smartphone Applikation wird ein Android Framework verwendet, 
während die Hardware aus einem Single Board Computer (SBC) bestehen soll. Somit wird dem Nutzer ein genauerer Einblick in sein Fahrstil verschafft. 
Mittels einem ODB-II zu Bluetooth Chip (ELM237) und Python, 
werden Daten aus dem Motormanagement sowie eigene Sensordaten, welche in einem selbstgemachtem Car-PC eingebaut sind,
gesammelt und aufbereitet. 
Die gesammelten Daten können dann von dem Smartphone aus in die Cloud geladen werden, welche dann durch unsere
Webapplikation genauer inspiziert werden können.
\clearpage % DO NOT REMOVE