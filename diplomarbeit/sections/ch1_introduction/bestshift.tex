\lfoot{Autoren: Fitim Faiku \& Hüseyin Bozkurt}
\section{Das Projekt BestShift}
\label{sec:projektbestshift}

Um die Probleme aus Kapitel 1 zu lösen wurde das Projekt BestShift entwickelt. Das allgemeine Ziel ist es einem Kraftfahrzeugfahrer während und nach der Fahrt Information anzubieten. Mittels der gesammelten Informationen werden unter anderem die drei Hauptfeatures, ein Schaltvorschlag (Sektion \ref{subsec:schaltvorschlag}), eine Verbrauchsanalyse (Sektion \ref{subsec:verbrauchsanalyse}), und eine Fahrgastkomfortanalyse (Sektion \ref{subsec:fahrgastkomfortanalyse}), implementiert. Diese Features sollen auch eine längerfristige Verbesserung des Fahrstils bewirken.

Das Projekt besteht aus unterschiedlichen Komponenten, die alle als ein System zusammenarbeiten (siehe Abbildung \ref{fig:konzept1}) und die drei Hauptfeatures realisieren. Diese Komponenten sind ein Car-PC (siehe Sektion \ref{subsec:carpc}) eine Android Applikation für die Darstellung während der Fahrt (siehe Sektion \ref{subsec:androidapp}) und eine Web-Applikation, welche für die retrospektive Darstellung und für Vergleiche zuständig ist (Sekton \ref{subsec:webapp}). Eine möglichst umweltschonende Implementierung war uns auch sehr wichtig.

\textbf{Projektmanagement\newline}
Die Entwicklung und konkrete Umsetzung des Systems war nicht von Anfang an klar. Über die Zeit, in welcher das Projekt umgesetzt wurde, haben sich immer wieder neue Komponenten hinzugefügt oder wurden gelöscht. Es mussten daher kurzfristige Änderungen schnell adaptiert werden. Glücklicherweise ist das Diplomprojekt BestShift aus der Idee eines Schülers entstanden, und es konnte daher der Rahmen der Umsetzung selber fest gelegt werden.

Aufgrund dessen haben wir uns für eine agile Projektmanagementmethode entschieden. Wir haben uns für SCRUM entschieden, da es zielführendes Arbeiten unterstützt und die inkrementelle Entwicklung von Software ermöglicht. Es ist damit auch sehr gut möglich, die Schwachstellen im Projekt schnell aufzudecken - was wesentlich für die Produktivität ist. Gezieltes Risikomanagement kann durch SCRUM ebenfalls betrieben werden. 

Unter agilem Projektmanagement, insbesondere SCRUM, gibt es den sogenannten
Sprint; ein zeitlich begrenzter Arbeitsaufwand. Ein Sprint wird im vorhinein geplant, wobei Wert auf den \textit{scope of work} (zu erledigende Arbeitspakete) gelegt wird. 
Zusätzlich dazu gibt es zum Beginn einer jeden
Arbeitsphase ein kleines Meeting (Daily Scrum Meeting). In diesem Meeting erklärt jedes Mitglied, was es in der letzten Arbeitsphase gearbeitet hat, was es in dieser Arbeitsphase tun wird und welche Probleme es beim Arbeiten gegeben hat.
Anders als bei der traditionellen Variante hat SCRUM den großen Vorteil, dass neue User Stories, und damit neue Funktionen der Software, on-the-go, also während das Projekt bereits läuft, hinzugefügt werden können. 
Grundsätzlich ist also SCRUM sinnvoll, wenn man mögliche Veränderungen erwartet. Jedoch sind die Grundelemente der erzeugten Software (und somit der Umfang der Userstories) sehr früh festgelegt.