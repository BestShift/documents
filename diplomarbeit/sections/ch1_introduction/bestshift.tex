\lfoot{Autor: Faiku Fitim}
\subsection{Das Projekt BestShift}

BestShift ist ein Projekt, welches dem Fahrer Informationen während- und nach der Fahrt bieten soll. 
Dazu werden in unterschiedlichen Teilen des Projektes 
diverse einzelne Komponenten einer hybriden Android Applikation und einer Hardware Schnittstelle für das ODB-II 
Interface entwickelt. 
Eine zusätzliche Web-App ermöglicht es, Fahrten retrospektiv zu betrachten und während der Fahrt gesammelte Daten innerhalb eines sozialen Netzwerkes zu teilen.
 

Für die Smartphone Applikation wird ein Android Framework verwendet, 
während die Hardware aus einem Single Board Computer (SBC) bestehen soll. 
Mittels einem ODB-II zu Bluetooth Chip (ELM237) und Python, 
werden Daten aus dem Motormanagement sowie eigene Sensordaten, welche in einem selbstgemachtem Car-PC eingebaut sind,
gesammelt und aufbereitet. 
Die gesammelten Daten können dann von dem Smartphone aus in die Cloud geladen werden, welche dann durch unsere
Webapplikation genauer inspiziert werden können.

In diesem Projekt wird eine Applikation für das mobile Betriebssystem Android implementiert, 
die dem Fahrer eines KFZ während der Fahrt Informationen zu Fahrgastbequemlichkeit und verbrauchseffizienter Fahrweise gibt. 
Dazu werden Daten aus dem Motormanagement verwendet. 
Für weitere benötigte Daten (z.B. Beschleunigungswerte) werden zusätzliche externe Sensoren in einem portablen CarPC integriert. 
Das Herzstück des Projektes ist aber die Webapplikation, die die Retrospektive Fahrtenanalyse und eine Social Media implentierung beinhaltet.
Diese Webapplikation wird die Möglichkeit bieten, die gesammelten Daten aller gespeicherten 
Fahrten für spätere Analysen graphisch einfach aufbereitet anzuzeigen. 
Die einzelnen Messwerte sollen dabei mit geographischen Informationen verknüpft werden, 
um dem Fahrer zum Beispiel zu zeigen, welche Stellen der Strecke besonders verbrauchsintensiv oder unbequem für den Fahrgast waren. 

Als weitere Funktion soll die Applikation aus den ermittelten Daten dem Fahrer den 
momentan am energieeffizientesten oder leistungsstärksten Gang vorschlagen können.
\clearpage % DO NOT REMOVE