\lfoot{Autor: Hüseyin Bozkurt}
\subsection{Projekt Management}

Da BestShift keiner konkreten Angabe unterlag,
mussten wir uns auf kurzfristige Änderungen gefasst machen, da die Herangehensweise der Implementierung der fahrzeugtechnischen Aspekte für uns noch nicht bekannt waren.
Das Team durfte über den Umfang des Projektes, bis auf kleine Vorgaben, selbst entscheiden.
Aufgrund dessen haben wir uns für eine agile Projektmanagementmethode entschieden. 
Wir haben uns für SCRUM entschieden, da es zielführendes Arbeiten unterstützt und die inkrementelle Entwicklung von Software ermöglicht. 
Ausserdem ist Transparenz ein sehr wichtiger Punkt, da unser Projektleiter immer auf dem aktuellsten Stand sein muss. Diese Transparenz ermöglicht es, Schwachstellen schnell aufzudecken und ist wesentlich für die Produktivität von SCRUM.
Gezieltes Risikomanagement kann durch SCRUM ebenfalls betrieben werden. 

\subsubsection{SCRUM}
Unter agilem Projektmanagement, insbesondere SCRUM, gibt es den sogenannten
Sprint; ein zeitlich Begrenzter Arbeitsaufwand. Ein Sprint wird logischerweise im vorhinein geplant, wobei man Wert auf den \textit{scope of work} (zu erledigende Arbeitspakete) legt. 
Zusätzlich dazu, gibt es zum Beginn einer jeden
Arbeitsphase ein kleines stand-up Meeting (Daily Scrum Meeting). In diesem Meeting erklärt jedes Mitglied was es in der letzten Arbeitsphase gearbeitet hat, was es in dieser Arbeitsphase tun wird und welche Probleme es beim Arbeiten gegeben hat.
Anders als bei der traditionellen Variante hat SCRUM den großen Vorteil, dass neue User Stories, und damit euch neue Funktionen der Software, on-the-go, also während das Projekt bereits läuft, hinzugefügt werden können. Grundsätzlich kann man sagen, dass SCRUM sinnvoll ist, wenn man mögliche Veränderungen anstrebt.

\subsubsection{Artefakte und Sprints}
Das Diplomprojekt BestShift wurde von Anfang an in 5 Sprints aufgeteilt, welche pünktlich vom Proejektleiter abgenommen wurden.
\begin{description}
\item Sprint 01: 18.Nov 2015 - 09.Dez 2015
\item Sprint 02: 09.Dez 2015 - 13.Jan 2016
\item Sprint 03: 13.Jan 2016 - 10.Feb 2016
\item Sprint 04: 10.Feb 2016 - 02.März 2016
\item Sprint 05: 02.März 2016 - 14.März 2016
\end{description}

Diese Sprints wurden durch die sogenannten Abnahmeprotokolle verifiziert.
Zu jedem Sprint gab es ebenfalls Sprintdokumentationen, die umschreibt was konkret während des Sprints passiert ist.


\clearpage % DO NOT REMOVE