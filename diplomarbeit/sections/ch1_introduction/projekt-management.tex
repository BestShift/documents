\lfoot{Autor: Hüseyin Bozkurt}
\subsection{Projekt Management}

Da BestShift keiner konkreten Angabe unterlag,
mussten wir uns auf Änderungen im letzten Moment gefasst machen. Das Team 
durfte über den Umfang des Projektes selbst entscheiden.
immer größer. Aufgrunddessen haben wir uns auf eine agile Projektmanagementmethode fixiert. 
Wir haben uns für SCRUM entschieden, da es Zielführendes Arbeiten unterstützt und die inkrementelle Entwicklung
von Software ermöglicht. Ausserdem ist Transparenz ein sehr wichtiger Punkt, da unser PL immer auf dem neuesten
Stand sein muss. Diese Transparenz ermöglicht es, Schwachstellen aufzudecken und ist für die produktivität von SCRUM bekannt.
Gezieltes Risikomanagement kann durch SCRUM ebenfalls betrieben werden. 


\subsubsection{SCRUM}
Unter agilem Projektmanagement, insbesondere Scrum, gibt es den sogenannten
Sprint; ein zeitlich Begrenzter Arbeitsaufwand. Ein Spirnt wird logischerweise
im Vorhinein geplant, wobei man Wert auf den ¨scope of work¨(zu erledigende
Arbeitspakete) legt. Zusätzlich dazu, gibt es zum Beginn einer jeden
Arbeitsphase ein kleines Meeting (Daily Scrum Meeting). In diesem Meeting
erklärt jedes Mitglied was es in der letzten Arbeitsphase gearbeitet hat, was es in
dieser Arbeitsphase tun wird und welche Probleme es beim Arbeiten gegeben
hat.
Anders als bei der traditionellen Variante hat Scrum den großen Vorteil das neue
User Stories, und damit euch neue Funktionen der Software, on-the-go hinzugefügt
werden können. Grundsätzlich kann man sagen, dass Scrum Sinnvoll ist,
wenn man mögliche Veränderungen anstrebt.

\subsubsection{Artefakte und Sprints}
Das Diplomprojekt BestShift wurde von Anfang an in 5 Sprints aufgeteilt, welche pünktlich
von PL abgenommen wurden.
Sprint 01 18.Nov 2015 - 09.Dez 2015
Sprint 02 09.Dez 2015 - 13.Jan 2016
Sprint 03 13.Jan 2016 - 10.Feb 2016
Sprint 04 10.Feb 2016 - 02.März 2016
Sprint 05 02.März 2016 - 14.März 2016

Diese Sprints wurden durch die sogenannten Abnahmeprotokolle verifiziert.
Zu jedem Sprint gab es ebenfalls Sprintdokumentationen, die umschrieben was konkret 
während des Sprints passiert ist.


\clearpage % DO NOT REMOVE