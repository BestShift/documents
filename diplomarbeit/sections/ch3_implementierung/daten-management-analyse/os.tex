\lfoot{Autor: Daniel Melichar}
\subsubsection{Betriebssystem am Car-PC}
\label{subsec:oscarpc}

Der Car-PC muss gewisse Software Anforderungen erfüllen damit korrekt auf diesem entwickelt werden kann. Dafür wurde die Rasbperry Pi Version (Raspbian) vom gängigen GNU/Linux OS Debian verwendet und die benötigte Software installiert.

Da die Hardware und Sensorik des Car-PC während der Laufzeit des Projekts stetig in Entwicklung war, musste eine Lösung gefunden werden, mittels welcher die Verwendbarkeit des Betriebssystem getestet werden kann. Hierfür wurde das Tool Vagrant \cite{MELD.CH3-vagrant.website} verwendet. Damit kann man virtuelle Maschinen einfach erstellen und verwalten. 

Für das Verwalten der virtuellen Maschinen wird ein so gennantes \textit{Vagrantfile} erstellt und in diesem definiert, wie die Maschine aussehen soll. Für unseren Fall, sieht dies so aus wie bei Listing \ref{vagrantfile}.

\begin{lstlisting}[language=Ruby, caption=Vagrantfile,numbers=left, xleftmargin=1cm, numberstyle=\tiny\noncopynumber, label={vagrantfile}]
# Create a forwarded port mapping which allows access to a specific port
# within the machine from a port on the host machine. In the example below 
# accessing "localhost:8080" will access port 80 on the guest machine.
config.vm.network :forwarded_port, host: 8000, guest: 8000
config.vm.network :forwarded_port, host: 9001, guest: 9001

# Create a private network, which allows host-only access to the machine
# using a specific IP.
config.vm.network "private_network", ip: "192.168.33.10"

# If true, then any SSH connections made will enable agent forwarding.
# Default value: false
	config.ssh.forward_agent = true

# Share an additional folder to the guest VM. The first argument is
# the path on the host to the actual folder. The second argument is
# the path on the guest to mount the folder. And the optional third
# argument is a set of non-required options. Here we specify the 
# scripts that will setup the OS
config.vm.synced_folder "./data", "/home/vagrant/data"
config.vm.synced_folder "./scripts/tools", "/home/vagrant/tools"
config.vm.synced_folder "./scripts/setup", "/home/vagrant/.setup"

# Provider-specific configuration so you can fine-tune various
# backing providers for Vagrant. These expose provider-specific options.
config.vm.provider "virtualbox" do |vb|
	# Boot with headless mode
	vb.gui = false
	vb.name = "BestShift CPOS"
	host = RbConfig::CONFIG['host_os']

	# Giving a quarter of system memory to VM and access to 
	# all available cpu cores
	if host =~ /darwin/
		cpus = `sysctl -n hw.ncpu`.to_i
		# sysctl returns Bytes, converting to MB...
		mem = `sysctl -n hw.memsize`.to_i / 1024 / 1024 / 4
	elsif host =~ /linux/
		cpus = `nproc`.to_i
		# meminfo returns KB, converting to MB...
		mem = `grep 'MemTotal' /proc/meminfo | sed -e 's/MemTotal://' -e 's/ kB//'`.to_i / 1024 / 4
	else
		# hardcoding values for windows...
		cpus = 2
		mem = 1024
	end

	vb.customize ["modifyvm", :id, "--memory", mem]
	vb.customize ["modifyvm", :id, "--cpus", cpus]
end

# Provisioning
config.vm.provision "shell" do |shell|
	vagrant_shell_scripts_configure(
		shell,
		File.join(ROOT_PATH, "scripts"),
		"provision.sh",	{}
	)
end
\end{lstlisting}

In den Zeilen 1-14 wird die Netzwerkstruktur der Maschine definiert. Durch die Verwendung von SSH \cite{MELD.CH3-ssh.website} und einem privaten Netzwerk ist es für das Team möglich einen einfachen Zugriff auf die Maschine zu bekommen.

In den Zeilen 15-22 werden so genannte \textit{shared/synced folder} \cite{MELD.CH3-vagrant.syncedFolders} definiert. Damit können gewisse Ordner im System des Administrators, bzw. der Person, welche das Vagrantfile verwendet, auf der virtuellen Maschine verwendet werden. Das bedeutet konkret, dass alle Dateien eines Ordners im Hostsystem (z.B. Windows, OS X) auf das Guest-System (also das OS auf dem Car-PC) in einem äquivalenten Ordner kopiert werden. Bei unserem Vagrantfile werden die Ordner gesynced, in welchen sich die Skripte zur Installation der Software befinden.

In den Zeilen 24-46 ist die Hostsystem spezifische Konfiguration zu finden. Hier ist zu beachten, dass wir mittels der kostenlosen Virtualisierungssoftware Oracle VirtualBox \cite{MELD.CH3-virtualbox.website} arbeiten - es sind aber auch andere Möglich \cite{MELD.CH3-vagrant.providers}. Wir definieren außerdem, dass wir die virtuelle Maschine ohne graphisches Interface starten, da wir nur über das CLI arbeiten. Ab der Zeile 32 werden die Hardware Spezifikationen der virtuellen Maschine beschrieben, diese sind abhängig vom Hostsystem. Allgemein werden 25\% des Arbeitsspeichers und alle CPU Kerne der Maschine zu verfügung gestellt. Diese Hardware Spezifikationen sind um einiges leistungsstärker als jene die dann letztendlich auf den Car-PC benötigt werden - sie wurden so hoch angesetzt um Leistungstests des Systems durchzuführen.

In den Zeilen 52-58 definieren wir die Skripte welche die benötigte Software installieren. In dem \textit{provision.sh} File werden dann alle weitern Skripte aufgerufen und ausgeführt. Die benötigte Software wurde durch die Evaluation des Teams ermittelt. Bei Listing \ref{provisionsh} kann ein Snippet betrachtet werden die zeigt welche Software installiert wird.

\begin{lstlisting}[language=bash, caption=provision.sh,label={provisionsh}]
pushd ~
mkdir -p logs
mkdir -p conf
popd

# Bootstrap
source ./scripts/setup-vars.sh
source ./scripts/setup-config.sh
source ./scripts/setup-apt.sh

# Setup and install services dependencies
source ./scripts/setup-postgresql.sh
source ./scripts/setup-redis.sh

# Setup and install python related dependencies
source ./scripts/setup-buildessential.sh
source ./scripts/setup-python.sh
source ./scripts/setup-java.sh

# Cleanup
source ./scripts/cleanup.sh
\end{lstlisting}

Alles im allem wurden Skripte geschrieben welche es ermöglichen den Car-PC, durch klicken eines einzigen Buttons, komplett aufzusetzen und für die weiteren Applikationen zur Verfügung zu stellen. Alle benötigten Vagrant Daten können auf \url{www.github.com/bestshift/vagrant} näher betrachtet werden.