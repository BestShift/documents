\lfoot{Autor: Daniel Melichar}
\subsubsection{Betriebssystem am Car-PC}

Der Car-PC muss gewisse Software Anforderungen erfüllen damit korrekt auf diesem Entwickelt werden kann. Dafür wurde die Rasbperry Pi Version (Raspbian) vom gängigen GNU/Linux OS Debian zur Hand genommen und die benötigte Software durch DevOps-Tools installiert.

Da die Hardware und Sensorik des Car-PC während der Laufzeit des Projekts stätig in Entwicklung war, musste eine Lösung gefunden werden, mittels welcher die Aufsetzung des Betriebssystem getestet werden kann. Hierfür wurde das Tool \textbf{Vagrant} verwendet. Damit kann man virtuelle Maschinen einfach erstellen und verwalten. 

Für das Verwalten der virtuellen Maschinen muss man ein so gennantes \textit{Vagrantfile} erstellen und in diesem definieren, wie die Maschine aussehen soll. Für unseren Fall, sieht dies so aus.

\begin{lstlisting}[language=Ruby, caption=Vagrantfile]
# Create a forwarded port mapping which allows access to a specific port
# within the machine from a port on the host machine. In the example below,
# accessing "localhost:8080" will access port 80 on the guest machine.
config.vm.network :forwarded_port, host: 8000, guest: 8000
config.vm.network :forwarded_port, host: 9001, guest: 9001

# Create a private network, which allows host-only access to the machine
# using a specific IP.
config.vm.network "private_network", ip: "192.168.33.10"

# If true, then any SSH connections made will enable agent forwarding.
# Default value: false
	config.ssh.forward_agent = true

# Share an additional folder to the guest VM. The first argument is
# the path on the host to the actual folder. The second argument is
# the path on the guest to mount the folder. And the optional third
# argument is a set of non-required options. Here we specify the 
# scripts that will setup the OS
config.vm.synced_folder "./data", "/home/vagrant/data"
config.vm.synced_folder "./scripts/tools", "/home/vagrant/tools"
config.vm.synced_folder "./scripts/setup", "/home/vagrant/.setup"

# Provider-specific configuration so you can fine-tune various
# backing providers for Vagrant. These expose provider-specific options.
config.vm.provider "virtualbox" do |vb|
	# Boot with headless mode
	vb.gui = false
	vb.name = "BestShift CPOS"
	host = RbConfig::CONFIG['host_os']

	# Giving a quarter of system memory to VM and access to 
	# all available cpu cores
	if host =~ /darwin/
		cpus = `sysctl -n hw.ncpu`.to_i
		# sysctl returns Bytes, converting to MB...
		mem = `sysctl -n hw.memsize`.to_i / 1024 / 1024 / 4
	elsif host =~ /linux/
		cpus = `nproc`.to_i
		# meminfo returns KB, converting to MB...
		mem = `grep 'MemTotal' /proc/meminfo | sed -e 's/MemTotal://' -e 's/ kB//'`.to_i / 1024 / 4
	else
		# hardcoding values for windows...
		cpus = 2
		mem = 1024
	end

	vb.customize ["modifyvm", :id, "--memory", mem]
	vb.customize ["modifyvm", :id, "--cpus", cpus]
end

# Provisioning
config.vm.provision "shell" do |shell|
	vagrant_shell_scripts_configure(
		shell,
		File.join(ROOT_PATH, "scripts"),
		"provision.sh",	{}
	)
end
\end{lstlisting}

Das Augenmerk sollte auf die letzten paar Zeilen gelegt werden - hier definieren wir die Skripte welche die benötigte Software installieren. In dem \textit{provision.sh} File werden dann alle weitern Skripte aufgerüfen und ausgeführt. Die benötigte Software wurde durch die Evaluation des Teams ermittelt. Im folgenden Snippet kann auf betrachtet werden, welche Software installiert wird.

\begin{lstlisting}[language=bash, caption=provision.sh]
pushd ~
mkdir -p logs
mkdir -p conf
popd

# Bootstrap
source ./scripts/setup-vars.sh
source ./scripts/setup-config.sh
source ./scripts/setup-apt.sh

# Setup and install services dependencies
source ./scripts/setup-postgresql.sh
source ./scripts/setup-redis.sh

# Setup and install python related dependencies
source ./scripts/setup-buildessential.sh
source ./scripts/setup-python.sh
source ./scripts/setup-java.sh

# Setup BestShift (to come)
#source ./scripts/setup-frontend.sh
#source ./scripts/setup-backend.sh

# Cleanup
source ./scripts/cleanup.sh
\end{lstlisting}

Alles im allem wurden Skripte geschrieben welche es ermöglichen den Car-PC, durch klicken eines einzigen Buttons, komplett aufzusetzen und für die weiteren Applikationen zur Verfügung zu stellen. Alle benötigten Vagrant Daten können auf \url{www.github.com/bestshift/vagrant} näher betrachtet werden.