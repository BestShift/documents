\lfoot{Autor: Daniel Melichar}
\subsubsection{Sammlung der Sensordaten}

Zunächst wurde für ein Programmierdesign Modell entwickelt in welchem beschrieben ist wie die Daten abgespeichert werden müssen. In diesem wird der Sensor als eigene Entität definiert, die eigentlichen Sensoren erben von ihr. Die Umsetzung kann hier begutachtet werden:

\begin{lstlisting}[language=Python, caption=Sensorenmodell]
#!/usr/bin/env python
class Gyroscope(models.Model):
    name = models.Attribute(required=True)
    created_at = models.DateTimeField(auto_now_add=True)
    xout = models.IntegerField(required=True)
    yout = models.IntegerField(required=True)
    zout = models.IntegerField(required=True)

    @property
    def xout_scaled(self):
        return self.xout/131

    @property
    def yout_scaled(self):
        return self.yout/131

    @property
    def zout_scaled(self):
        return self.zout/131

    def get_all(self):
        return [self.xout, self.xout_scaled, self.yout, self.yout_scaled, self.zout, self.zout_scaled]

class Temperature(models.Model):
    name = models.Attribute(required=True)
    created_at = models.DateTimeField(auto_now_add=True)
    temp = models.FloatField(required=True)

class Accelerometer(models.Model):
    name = models.Attribute(required=True)
    created_at = models.DateTimeField(auto_now_add=True)
    xout = models.IntegerField(required=True)
    yout = models.IntegerField(required=True)
    zout = models.IntegerField(required=True)
    xrot = models.IntegerField()
    yrot = models.IntegerField()

    @property
    def xout_scaled(self):
        return self.xout/131

    @property
    def yout_scaled(self):
        return self.yout/131

    @property
    def zout_scaled(self):
        return self.zout/131

    def get_all(self):
        return [self.xout, self.xout_scaled, self.yout, self.yout_scaled, self.zout, self.zout_scaled, self.xrot, self.yrot]
\end{lstlisting}

Wie genau wir Daten erhalten wird hier beschrieben \todo{MELD: Referenz}. Der Car-PC musste aber auch so eingerichtet werden, so nach Bootende bereits anfängz zu lesen. Hierfür wurden so Service geschrieben (Bash Skripte im \lstinline{/etc/init.d/} Verzeichnis die sich automatisch ausführen) welche dann wiederum jene Python Skripte startet die für das Auslesen und Speichern (in die Redis Datenbank) durchführen. Anbei befindet sich das für den Beschleunigungssensor zuständigen Skript:

\begin{lstlisting}[language=Bash, caption=Beschleunigungssensor Service]
#!/bin/sh
# /etc/init.d/bsdata-accel

### BEGIN INIT INFO
# Provides:          bsdata-accel
# Required-Start:    $remote_fs $syslog
# Required-Stop:     $remote_fs $syslog
# Default-Start:     2 3 4 5
# Default-Stop:      0 1 6
# Short-Description: Simple script to start a program at boot
# Description:       A simple script from  which will start / stop a program a boot / shutdown.
### END INIT INFO

# If you want a command to always run, put it here

# Carry out specific functions when asked to by the system
case "$1" in
  start)
    echo "Starting bsdata-accel"
    # run application you want to start
    /usr/bin/python /usr/local/bin/get_data_accel.py
    ;;
  stop)
    echo "Stopping bdsata-accel"
    # kill application you want to stop
    killall bsdata-accel
    ;;
  *)
    echo "Usage: /etc/init.d/bsdata-accel {start|stop}"
    exit 1
    ;;
esac

exit 0
\end{lstlisting}

\clearpage
Eines der größten Probleme die wir (Tobias Perny und Daniel Melichar) hatten war ein, rückblickend Betrachtet, banaler Fehler. Das Problem war, dass das Service an sich funktioniert. Wir haben getestet ob es sich starten lässt und ob es die Sensordaten in Redis gespeichert werden. Dies hat zwar prinzipiell funktioniert, aber nach einem Reboot des Car-PCs mussten die Sensorservice immer wieder neu gestartet werden. Uns war eine lange Zeit nicht bewusst wie dies Problem entsteht. Letztendlich sind wir dann durch einen Witz auf das Problem gekommen: \textit{„Hast du den Redis-Server überhaupt gestartet?“}. Danach gingen wir nochmals die Dokumentation durch und fanden heraus, dass Redis einen eigenen Service benötigt um den Server zu starten. Dies sah dann letztendlich wie folgt aus:

\begin{lstlisting}[language=Bash, caption=Redis Server Service]
#! /bin/sh
### BEGIN INIT INFO
# Provides:   redis-server
# Required-Start: $syslog $remote_fs
# Required-Stop:  $syslog $remote_fs
# Should-Start:   $local_fs
# Should-Stop:    $local_fs
# Default-Start:  2 3 4 5
# Default-Stop:   0 1 6
# Short-Description:  redis-server - Persistent key-value db
# Description:    redis-server - Persistent key-value db
### END INIT INFO


PATH=/usr/local/sbin:/usr/local/bin:/sbin:/bin:/usr/sbin:/usr/bin
DAEMON=/usr/bin/redis-server
DAEMON_ARGS=/etc/redis/redis.conf
NAME=redis-server
DESC=redis-server

RUNDIR=/var/run/redis
PIDFILE=$RUNDIR/redis-server.pid

test -x $DAEMON || exit 0

if [ -r /etc/default/$NAME ]
then
  . /etc/default/$NAME
fi

. /lib/lsb/init-functions

set -e

case "$1" in
  start)
      echo -n "Starting $DESC: "
      mkdir -p $RUNDIR
      touch $PIDFILE
      chown redis:redis $RUNDIR $PIDFILE
      chmod 755 $RUNDIR

      if [ -n "$ULIMIT" ]
      then
        ulimit -n $ULIMIT
      fi

      if start-stop-daemon --start --quiet --umask 007 --pidfile $PIDFILE --chuid redis:redis --exec $DAEMON -- $DAEMON_ARGS
      then
        echo "$NAME."
      else
        echo "failed"
      fi
  ;;
  
  stop)
      echo -n "Stopping $DESC: "
      if start-stop-daemon --stop --retry forever/TERM/1 --quiet --oknodo --pidfile $PIDFILE --exec $DAEMON
      then
        echo "$NAME."
      else
        echo "failed"
      fi
      rm -f $PIDFILE
      sleep 1
  ;;

  restart|force-reload)
      ${0} stop
      ${0} start
  ;;

  status)
      status_of_proc -p ${PIDFILE} ${DAEMON} ${NAME}
  ;;

  *)
      echo "Usage: /etc/init.d/$NAME {start|stop|restart|force-reload|status}" >&2
      exit 1
  ;;
esac

exit 0
\end{lstlisting}