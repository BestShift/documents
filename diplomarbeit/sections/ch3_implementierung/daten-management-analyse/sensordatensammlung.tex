\lfoot{Autor: Daniel Melichar}
\subsubsection{Sammlung der Sensordaten}

Zunächst wurde ein Datenmodell entwickelt in welchem beschrieben ist, wie die Daten abgespeichert werden müssen. In diesem wird der Sensor als eigene Entität definiert, die eigentlichen Sensoren erben von ihr. Die Umsetzung zeigt Listing \ref{lst:sensorsammlung}

\begin{lstlisting}[style=pythonstyle, caption=Sensorenmodell, label={lst:sensorsammlung}]
#!/usr/bin/env python

# Die Umsetzung dieses Model ist basieren auf der Python Libary Redisco.
# Diese erlaubt es Klassenobjekte zu erstellen, welche dann automatisch in 
# die Datenbank gespeichert werden.
# Redisco Library: https://github.com/kiddouk/redisco

# Basis Klasse fuer den Gyroskop Sensor welche die fuer uns relevanten 
# Daten eines Gyroskop Sensors beschreibt. Die konkreten Werte beziehen
# sich auf die Werte in allen drei Achsen, dem Erstellungsdatum der Klasse 
# (aquivalent zum Zeitpunkt des Auslesens der Werte), und einem Namen der 
# als ID dient.
# 
# Die *_scaled Methoden der Achsen Rechnen den konkreten Wert der Binaerdaten aus.
# 
class Gyroscope(models.Model):
    name = models.Attribute(required=True)
    created_at = models.DateTimeField(auto_now_add=True)
    xout = models.IntegerField(required=True)
    yout = models.IntegerField(required=True)
    zout = models.IntegerField(required=True)

    @property
    def xout_scaled(self):
        return self.xout/131

    @property
    def yout_scaled(self):
        return self.yout/131

    @property
    def zout_scaled(self):
        return self.zout/131

    # Liefert alle aktuellen Daten zurueck
    def get_all(self):
        return [self.xout, self.xout_scaled, self.yout, self.yout_scaled, self.zout, self.zout_scaled]

# Basis Klasse fuerr den Temperatursensor welche sich hauptsaechlich
# auf die Raumtemperatur fokusiert. Zusaetzlich dazu muss noch ein 
# Name als ID und Erzeugungsdatum/Sensorlesezeitpunkt angegeben werden.
# 
class Temperature(models.Model):
    name = models.Attribute(required=True)
    created_at = models.DateTimeField(auto_now_add=True)
    temp = models.FloatField(required=True)


# Basis Klasse fuer den Beschleunigungssensor welche die relevanten 
# Information beschreibt. Auch hier muss ein Name als ID und ein 
# Erzeugungsdatum/Sensorlesezeitpunkt angegeben werden. Weiters werden
# die konkreten Werte, also jene der drei Achsen sowie die Rotation, bei
# Angabe in die Datenbank gespeichert.
# 
# Die *_scaled Methoden der Achsen Rechnen den konkreten Wert der Binaerdaten aus.
#
class Accelerometer(models.Model):
    name = models.Attribute(required=True)
    created_at = models.DateTimeField(auto_now_add=True)
    xout = models.IntegerField(required=True)
    yout = models.IntegerField(required=True)
    zout = models.IntegerField(required=True)
    xrot = models.IntegerField()
    yrot = models.IntegerField()

    @property
    def xout_scaled(self):
        return self.xout/131

    @property
    def yout_scaled(self):
        return self.yout/131

    @property
    def zout_scaled(self):
        return self.zout/131

    # Liefert alle aktuellen Daten zurueck
    def get_all(self):
        return [self.xout, self.xout_scaled, self.yout, self.yout_scaled, self.zout, self.zout_scaled, self.xrot, self.yrot]
\end{lstlisting}

Der Car-PC musste so eingerichtet werden, so dass nach Ende des Boot-Vorgange bereits anfängt zu lesen. Hierfür wurden Init-Skripte, also Skripte die beim Startvorgang ausgeführt werden, programmiert. Diese verwenden dann wiederum das oben angeführte Model um das Auslesen und Speichern (in die Redis Datenbank) durchführen. Listing \ref{lst:init} zeigt das für den Beschleunigungssensor zuständigen Skript:

\begin{lstlisting}[language=Bash, caption=Beschleunigungssensor Service, label={lst:init}]
#!/bin/sh
# /etc/init.d/bsdata-accel

### BEGIN INIT INFO
# Provides:          bsdata-accel
# Required-Start:    $remote_fs $syslog
# Required-Stop:     $remote_fs $syslog
# Default-Start:     2 3 4 5
# Default-Stop:      0 1 6
# Short-Description: Simple script to start a program at boot
# Description:       A simple script from  which will start / stop a program a boot / shutdown.
### END INIT INFO

# If you want a command to always run, put it here

# Carry out specific functions when asked to by the system
case "$1" in
  start)
    echo "Starting bsdata-accel"
    # run application you want to start
    /usr/bin/python /usr/local/bin/get_data_accel.py
    ;;
  stop)
    echo "Stopping bdsata-accel"
    # kill application you want to stop
    killall bsdata-accel
    ;;
  *)
    echo "Usage: /etc/init.d/bsdata-accel {start|stop}"
    exit 1
    ;;
esac

exit 0
\end{lstlisting}

Für den automatisierten Start des Redis-Services wird ebenfalls ein Init-Skript benötigt (Listing \ref{lst:redisinit}).

\begin{lstlisting}[language=Bash, caption=Redis Server Service, label={lst:redisinit}]
#! /bin/sh
### BEGIN INIT INFO
# Provides:   redis-server
# Required-Start: $syslog $remote_fs
# Required-Stop:  $syslog $remote_fs
# Should-Start:   $local_fs
# Should-Stop:    $local_fs
# Default-Start:  2 3 4 5
# Default-Stop:   0 1 6
# Short-Description:  redis-server - Persistent key-value db
# Description:    redis-server - Persistent key-value db
### END INIT INFO

# Laufzeitvariablen
PATH=/usr/local/sbin:/usr/local/bin:/sbin:/bin:/usr/sbin:/usr/bin
DAEMON=/usr/bin/redis-server
DAEMON_ARGS=/etc/redis/redis.conf
NAME=redis-server
DESC=redis-server

# Redis Ordner
RUNDIR=/var/run/redis
PIDFILE=$RUNDIR/redis-server.pid

# Testen ob der Redis service laueft
test -x $DAEMON || exit 0

if [ -r /etc/default/$NAME ]
then
  . /etc/default/$NAME
fi

. /lib/lsb/init-functions

set -e

case "$1" in
  start)
      echo -n "Starting $DESC: "
      mkdir -p $RUNDIR
      touch $PIDFILE
      chown redis:redis $RUNDIR $PIDFILE
      chmod 755 $RUNDIR

      if [ -n "$ULIMIT" ]
      then
        ulimit -n $ULIMIT
      fi

      if start-stop-daemon --start --quiet --umask 007 --pidfile $PIDFILE --chuid redis:redis --exec $DAEMON -- $DAEMON_ARGS
      then
        echo "$NAME."
      else
        echo "failed"
      fi
  ;;
  
  stop)
      echo -n "Stopping $DESC: "
      if start-stop-daemon --stop --retry forever/TERM/1 --quiet --oknodo --pidfile $PIDFILE --exec $DAEMON
      then
        echo "$NAME."
      else
        echo "failed"
      fi
      rm -f $PIDFILE
      sleep 1
  ;;

  restart|force-reload)
      ${0} stop
      ${0} start
  ;;

  status)
      status_of_proc -p ${PIDFILE} ${DAEMON} ${NAME}
  ;;

  *)
      echo "Usage: /etc/init.d/$NAME {start|stop|restart|force-reload|status}" >&2
      exit 1
  ;;
esac

exit 0
\end{lstlisting}