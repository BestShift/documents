\subsubsection{Entwicklungsumgebung und Code Guidelines}
\lfoot{Autor: Daniel Melichar}
Für den weiteren Verlauf der Web-Applikation musste eine effiziente Umgebung erstellt werden, welche die Entwickler bestmöglich unterstüzt. Die Umgebung musste automatisiert, nachhaltig, einfach zu verstehen und konfigurierbar sein. Zusätzlich dazu, mussten Standards für den Code gesetzt werden, damit alle auf die selbe Art und Weise programmieren, um eine schnelle Umsetzung zu ermöglichen. Das Ziel dieses Style Guides ist, eine Sammlung von Best Practices und Gestaltungsrichtlinien für AngularJS-Anwendungen aufzuzeigen. Sie wurden aus den folgenden Quellen zusammengestellt:

\begin{enumerate}
\def\labelenumi{\arabic{enumi}.}
\item AngularJS Dokumentation
\item Quelltexte oder Artikel
\item Erfahrung des Teams und Lehrer
\end{enumerate}

Für allgemeine Richtlinien der JavaScript-Entwicklung sollte man folgende Guides begutachten:

\begin{enumerate}
\def\labelenumi{\arabic{enumi}.}
\item \href{http://google-styleguide.googlecode.com/svn/trunk/javascriptguide.xml}{Googles JavaScript-Style-Guide (empfohlen)}
\item \href{https://developer.mozilla.org/en-US/docs/Developer_Guide/Coding_Style}{Mozillas JavaScript-Style-Guide}
\item \href{https://github.com/styleguide/javascript}{GitHubs JavaScript-Style-Guide}
\item \href{http://javascript.crockford.com/code.html}{Douglas Crockfords JavaScript-Style-Guide}
\item \href{https://github.com/airbnb/javascript}{Airbnb JavaScript-Style-Guide}
\end{enumerate}

Die Umsetzung einer Applikation für das Web kann sehr schnell komplex werden. Mit einigen Development-Tools soll die Komplexität der Umsetzung auf ein absolutes Minimum reduziert werden und das Team damit möglichst effizient arbeiten können. Da eine große AngularJS-Anwendung viele Komponenten hat, sollten diese mit Hilfe einer Verzeichnishierarchie strukturiert werden. Auf einer oberen Ebene eine Aufteilung nach Art der Komponenten und auf einer tieferen Ebene eine Aufteilung nach Funktionalität. Die verwendeten Tools der Web-Applikation sind:

\begin{enumerate}
\item \textbf{GitHub} (\url{http://github.com}) ist ein Versionierungstool mit welchem die Problematik der Zusammenarbeit von mehreren Personen auf den selben Dateien gelöst wird.
\item \textbf{Grunt} (\url{http://gruntjs.com}) ist ein so gennanter \textit{Task-Manager}. Mittels diesem Tool können gewisse Operationen automatisiert durchgeführt werden - z.B. Rechtschreibprüfung, starten eines lokalen Servers, minimierung von Code auf das Notwendigste.
\item \textbf{Bower} (\url{http://bower.io}) ist ein so gennantes \textit{Dependency-Management-Tool}. Prinzipiell ermöglichst es, die benötigte Software zu verwalten, installieren und automatisieren.
\end{enumerate}

\todo{MELD: From here, redo}

\clearpage

\textbf{Markup\newline}
Der HTML-Markup ist wichtig (siehe Sektion \ref{html}) und sollte in einem Team so geschrieben werden, als sei sie von derselben Person. Für die Implementierung mittels der AngularJS Templates sollte folgende Struktur in Betracht gezogen werden.

\begin{lstlisting}
<!DOCTYPE html>
<html lang="en">
<head>
  <meta charset="utf-8">
  <title>Meine App</title>
</head>
<body>
  <div ng-app="myApp">
    <div ng-view></div>
  </div>
  <script src="angular.js"></script>
  <script src="app.js"></script>
</body>
</html>
\end{lstlisting}


Um den Code nicht unnötig zu verkomplizieren, fügt man AngularJS-spezifische Direktiven hinter Standard-Attributen ein. Dadurch ist es einfacher, sich den Code anzusehen und durch das Framework erweitertes HTML zu erkennen (was die Wartbarkeit verbessert).

\begin{lstlisting}
<form class="frm" ng-submit="login.authenticate()">
<div>
<input class="ipt" type="text" placeholder="name" require ng-model="user.name">
</div>
</form>
\end{lstlisting}


\textbf{Namensgebung\newline}

Für die einzelnen Referenzen gibt es Vorschriften, welche beschreiben, wie die Referenz aussehen soll.

\begin{tabular}{|l|l|l|l|}
\hline
\textbf{Element} & \textbf{Style} & \textbf{Beispiel} & \textbf{Verwendung bei} \\ \hline
Module & lowerCamelCase & angularApp &  \\ \hline
Controller & funktionalität + Ctrl & adminCtrl &  \\ \hline
Direktiven & lowerCamelCase & userInfo &  \\ \hline
Filter & lowerCamelCase & userFilter &  \\ \hline
Services & UpperCamelCase & User & Konstruktor \\ \hline
Services & lowerCamelCase & dataFactory & sonstige \\ \hline
\end{tabular}

\textbf{Module\newline}
Module sollten in \textit{lowerCamelCase} benannt werden. Um deutlich zu machen, dass das Modul \textit{b} ein Untermodul von \textit{a} ist, können sie durch Namespaces verschachteln werden, z. B.: \textit{a.b}.

Es gibt zwei verbreitete Wege, nach denen Module strukturiert werden können: Nach Funktionalität
oder nach Typ der Komponente. Derzeit gibt es keinen großen Unterschied, aber die erste Variante sieht sauberer aus. Außerdem wird - wenn lazy-loading für die Module implementiert ist - die Performance der verbessert.

\textbf{Services\newline}
Dieser Abschnitt enthält Informationen über die AngularJS Service-Komponente. Er bezieht sich nicht auf eine spezielle Definitionsweise (d.h. als Provider, Factory oder Service), falls nicht ausdrücklich genannt.

Services in camelCase benennen, bzw UpperCamelCase (PascalCase), um Services zu benennen, die als Konstruktoren verwendet werden, d.h.:

\begin{lstlisting}[language=JavaScript]
module.controller('MainCtrl', function ($scope, User) {
  $scope.user = new User('foo', 42);
});

module.factory('User', function () {
  return function User(name, age) {
    this.name = name;
    this.age = age;
  };
});
\end{lstlisting}

Alle anderen Service in \textit{lowerCamelCase} benennen. Kapslung der gesamte Anwendungslogik in Services beachten. Services, die eine bestimmte Domäne abbilden, sollten bevorzugt als \textit{service} statt als \textit{factory} geschrieben werden. Auf diese Weise können die Vorteile der klassischen Vererbung einfacher genutzt werden:

\begin{lstlisting}[language=JavaScript]
function Human() {
  // body
}
Human.prototype.talk = function() {
  return "I'm talking";
};

function Developer() {
  // body
}
Developer.prototype = Object.create(Human.prototype);
Developer.prototype.code = function() {
  return "I'm coding";
};

myModule.service('Human', Human);
myModule.service('Developer', Developer);
\end{lstlisting}

Für einen sitzungsbezogenen Cache kannst du \textit{\$cacheFactory} verwenden. Diesen sollte man nutzen, um die Ergebnisse von Anfragen oder aufwändigen Berechnungen zwischenzuspeichern.Falls ein Service konfiguriert werden muss, definiert man ihn als Provider und konfiguriert im \textit{config}-Callback, wie hier:

\begin{lstlisting}[language=JavaScript]
angular.module('demo', [])
.config(function ($provide) {
  $provide.provider('sample', function () {
    var foo = 42;
    return {
      setFoo: function (f) {
        foo = f;
      },
      $get: function () {
        return {
          foo: foo
        };
      }
    };
  });
});

var demo = angular.module('demo');

demo.config(function (sampleProvider) {
  sampleProvider.setFoo(41);
});
\end{lstlisting}