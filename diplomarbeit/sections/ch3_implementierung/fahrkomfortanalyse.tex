\lfoot{Autor: Fitim Faiku}
\subsection{Fahrkomfortanalyse}

Da die bewertung der einzelnen Fahrer durch den Fahrgast erfolgt ist jedem klar.
Diese Bewertung faellt meistens aus den Faktoren wie stark ein Fahrer beschleunigt und in welcher Geschwindigkeit er um eine Kurve faehrt.
Um einen KFZ-Einsteiger mit mueglichst Informativen Daten zu \"bewerten\" habe ich mir einen Kammschen Kreis als Hilfe genommen.

\subsubsection{Kammscher Kreis }
Ein Kammsher Kreis ist ein ein Kreis in dem durch Polarkoordinaten die Extremwerte einer Fahrt eingezeichnet sind.
Der Kreis ist so aufgebaut, dass in der positiven y-Achse alle Bremskraefte eingezeichnet werden, negative y-Achse alle beschleunigen,
negative x-Achse alle rechts Kurven, positiven x-Achse alle Linkskurven.
Es ist also alles verkehrt.
\todo{Kammscher Kreis Foto}

\subsubsection{Implementierung}
Um diesen Kammschen Kreis richtig darzustellen habe ich mir die Klasse canvas als Hilfe genommen, wobei die von Graphics erbt.
Mittels Canvas konnte ich zufaellig generierte 0-360grad Winkel und Radien erstellen und die dann in Polarkoordinaten umrechnen.
Somit konnte ich alle generierten werte Darstellen. 

public int runden(float roh){
	float anzahl= round(roh/5);
	System.out.println("Gerundet:" + anzahl);
	int returnzahl=0;
	returnzahl= (int) (anzahl*5);
	return returnzahl;
}

Bei der Methode Runden nehme ich alle Winkel auf und Runde sie auf oder ab in 5er Schritten.



\clearpage % DO NOT REMOVE