\lfoot{Autor: Fitim Faiku}
\subsection{Fahrkomfortanalyse}

Dass die Bewertung der einzelnen Fahrer durch den Fahrgast erfolgt ist jedem klar.
Diese Bewertung wird meist durch gewisse Faktoren bestimmt. Diese können wie stark ein Fahrer beschleunigt und welche Kurvengeschwindigkeit ein Fahrer anwendet.
Um einen KFZ-Einsteiger mit möglichst informativen Daten zu \"bewerten\" habe ich mir einen Kamm'schen Kreis zur Hilfe genommen.

\subsubsection{Kammscher Kreis}
Ein Kamm'scher Kreis ist ein ein Kreis in dem durch Polarkoordinaten die Extremwerte einer Fahrt eingezeichnet sind.
Der Kreis ist so aufgebaut, dass in der positiven vertikalen Achse alle Bremskräfte, in der negativen vertikalen Achse alle Beschleunigungskräfte,
in der negative horizontalen Achse alle rechts Kurven, in der positiven horizontalen Achse alle Linkskurven eingezeichnet werden. \comment{SIMR: vllt. mit Himmelsrichtungen statt positiv und negativ + welche Achse}
Es sind die eingezeichneten Werte also spiegelverkehrt zur Intuition.

\subsubsection{Implementierung}
Um diesen Kamm'schen Kreis richtig darzustellen habe ich mir die Klasse \textit{Canvas} als Hilfe genommen, wobei diese von \textit{Graphics} erbt.
Mittels Canvas konnte ich zufällig generierte 0-360grad Winkel und Radien erstellen und die dann in Polarkoordinaten umrechnen.
Somit konnte ich alle generierten Werte darstellen. 

\lstinputlisting[caption=Runden-Methode, style=javastyle]{code/Runden.java}

Bei der Methode Runden nehme ich alle Winkel auf und runde sie auf oder ab in 5er Schritten.



\clearpage % DO NOT REMOVE