\lfoot{Autor: Fitim Faiku}
\subsection{Android App}

Die android App wir in Java erstellt wobei die Programmieroberflaeche Android Studio verwendet wird.
Zu jeder Klasse wird ein XML layout als View angesetzt und somit besteht eine Verbindung zwischen Code und Layout.

\subsubsection{Code-Dokumetation}
\lstinputlisting[caption=MyActivity, style=javastyle]{code/MyActivity.java}
 Um die Grafiken bestmoeglich darzustellen und dabei ein weites Spektrum an Usern zu erzielen, haben wir uns entschieden eine Android App zu erstellen.
 Dabei haben wir besonders auf die Benutzerfreundlichkeit der App geachtet.
 Die wichtigsten Kriterien dabei waren, dass es den Fahrer nicht zu viel an Konzentration abverlangt, dass er sehr leicht zwischen den einzelnen Feauters wechseln kann und das Design sehr Informant ist.
 
 Um die App zu realisieren habe ich mehrere Fragments erstellt zwischen denen man hin und her wischen kann.
 


 Es wird erstmal eine MainActivity Klasse erstellt, welche von FragementActivity erbt und von ActionTabListener implementieret, um die Methoden onTabReselected, onTabselected und onTabunselected zu Ueberschreiben.   
 
            
Nachher setzen wir die View der Klasse auf main mit:  
            
setContentView(R.layout.main);

Wir erstellen ein Objekt von ActionBar und fuegen mehrere Tabs hinzu
Wir setzen das Layout fuer die Navigation der Tabs mit
 

$actionbar.setNavigationMode(ActionBar.NAVIGATION_MODE_TABS);$

Nachher werden  mehrere Tabs hinzugefuegt mit:

actionbar.addTab(actionbar.newTab().setTabListener(this).setIcon(R.drawable.verbrauchwolke));

actionbar.addTab(actionbar.newTab().setTabListener(this).setIcon(R.drawable.komfort), true);

// Setzen des Tabs auf angeklickt mit True

actionbar.addTab(actionbar.newTab().setTabListener(this).setIcon(R.drawable.schaltvorschlag));

// Weiters werden Icons zu den Tabs hinzugefuegt.


Nun gehen wir rueber zu unserem Design unter main.xml
\lstinputlisting[caption=Design main, style=xmlstyle]{code/main.xml}
Wir erstellen ein ViewPager Layout welches das Wischen erlaubt mit:

<android.support.v4.view.ViewPager> </android.support.v4.view.ViewPager>

Wir Uebergeben dem Layout ebenfalls eine id auf die wir von unserer main Activity zugreifen: 

android:id="@+id/pager"



Nun erstellen wir eine Fragementadapter Klasse um die einzelnen Tabs den Klassen zuzuteilen. 
Die Klasse hat einen Kronstruktor welcher von der superklasse FragementManager Objekt Uebergeben bekommt.
Weiters hat die Klasse eine getIem Methode mit dem Input eines int Wertes. 
\lstinputlisting[caption=PageAdapter, style=javastyle]{code/FragementPageAdapter.java}

Hier wird nur geprueft welcher Tab angecklickt wurde dem Teilt er eine Klasse auf welche eine eigene View hat.
Sie hat noch eine getCount() Methode welches die Anzahl der Tabs zurueckgibt in unserem Fall 3.


Wir erstellen auf der Main Klasse zwei Objekte 

ViewPager viewpager;

FragmentPagerAdapter ft;

viewpager =(ViewPager) findViewById(R.id.pager);

ft = new FragementPageAdapter(getSupportFragmentManager());

viewpager.setAdapter(ft);

Das wird gemacht um zwischen den Oberflaechen wechseln zu koennen.

 
\subsubsection{MPAndroidchart als Tool}
Um die zwei Grafiken für den Emissionsausshuss und die Längsbeschleunigung darzustellen, haben wir uns für ein Framework entschieden welches und diese Aufgabe erleichterte.

 \begin{table}[!htb]
 	\centering
 	\caption{Vergleich der Charting-Frameworks}
 	\label{comparisonCharting}
 	\resizebox{\columnwidth}{!}{%
 		\begin{tabular}{l|c|c|c|l|}
 			\cline{2-5}
 			& \begin{tabular}[c]{@{}c@{}}MPAndorid\\ Chart\end{tabular} & \begin{tabular}[c]{@{}c@{}}AChart\\ Engine\end{tabular} & HelloChart & ErklÀrung \\ \hline
 			\multicolumn{1}{|l|}{Performance} & Gut & Gut & Gut &  \\ \hline
 			\multicolumn{1}{|l|}{\begin{tabular}[c]{@{}l@{}}Supported \\ Languages\end{tabular}} & \begin{tabular}[c]{@{}c@{}}iOS 7,8;\\ Java\end{tabular} & Java & Java & \begin{tabular}[c]{@{}l@{}}MPAC kann auch\\ weiterverwendet werden\\ fÃŒr eine iOS App\end{tabular} \\ \hline
 			\multicolumn{1}{|l|}{Documentation} & 85\% & 70\% & 50\% & \begin{tabular}[c]{@{}l@{}}MPAC hat eine sehr gute\\ in Kapiteln aufgelistete\\ Dokumentation, Tutorials\end{tabular} \\ \hline
 			\multicolumn{1}{|l|}{Recent activity} & 19.10.2015 & 26.07.2013 & 5.10.2015 & \begin{tabular}[c]{@{}l@{}}MPAC ist sehr aktuell\\ und bringt immer wieder \\ neue Bugfixes raus.\\ HelloChart ebenso.\end{tabular} \\ \hline
 			\multicolumn{1}{|l|}{Community} & 80\% & 60\% & 20\% & \begin{tabular}[c]{@{}l@{}}Bei MPAC und \\ AChartEngine wird eine\\ Antwort innerhalb von 7 \\ Tagen geliefert.\\ Eine Community bei Hello\\ Chart ist kaum vorhanden.\end{tabular} \\ \hline
 			\multicolumn{1}{|l|}{Usage examples} & 85\% & 70\% & 15\% & \begin{tabular}[c]{@{}l@{}}MPAC hat fÃŒr jeden Chart\\ ein eigenes Beispiel oder \\ auch Tutorials auf YouTube.\\ AChartEngine weniger \\ verwendbare Codebeispiele.\end{tabular} \\ \hline
 		\end{tabular}
 	}
 \end{table}

Im Großen und Ganzen ist MPAndroidChart das Beste was man wählen kann, da es eins der besten Dokumentationen hat, weiterhin hat es auch eine sehr gute Community, sehr viel gut gecodete und dokumentierte Beispiele und ist auch fuer Erweiterung gut, da die App auch auf einem IPhone aufgesetzt werden kann.
 
AChartEngine hat eine gute Dokumentation doch eine eher schwaechere Community und ist leider nicht sehr aktuell. 
HelloChart hat eins der schlechtesten Dokumentationen, gar keine Community und sehr wenig gecodete Beispiele, sie sind aber sehr Aktuell wenn es um die Versionen geht .

\subsubsection*{Implementierung von MPAndroidChart}
Als naechstes habe ich mithile der MPAndroidchart Library einige Graphen erstellt und sie mit Daten befuellt 

\lstinputlisting[caption=Liniendiagramm Beispiel, style=javastyle]{code/Beschleunigungskraefte.java}


 






\clearpage % DO NOT REMOVE