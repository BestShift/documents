\usepackage[in]{fullpage}
\usepackage[T1]{fontenc}
\usepackage[utf8]{inputenc}
\usepackage[ngerman]{babel}
\usepackage{graphicx} 
\usepackage{textcomp}
\usepackage{sectsty}
\usepackage{caption}
\usepackage{array}
\usepackage{colortbl}
\usepackage{footmisc}
\usepackage{fancyhdr}
\usepackage{ccicons}
\usepackage{suffix}
\usepackage{multirow}
\usepackage{tabularx}
\usepackage{color}
\usepackage{url}
\usepackage{setspace}
\usepackage{enumitem}
\usepackage{listings}
\usepackage{xcolor}
\usepackage{filecontents}
\usepackage{glossaries}
\usepackage{titlesec}
\usepackage{mhchem}
\usepackage[headsep=1cm,headheight=3cm,hmargin=2cm,vmargin=2.5cm]{geometry}
\usepackage[nolist]{acronym}
\usepackage[toc,page]{appendix}

\usepackage[
	colorlinks,
	citecolor=black,
	filecolor=black,
	linkcolor=black,
	urlcolor=black,
	linktoc=all
]{hyperref}

%%%%%%%%%%%%%%%%%%%%%%%%%%%%%%%%%%%%%%%%%%%%%%%%%%%%%%%%%%%%%%%%
%% Useful commands

% Comments and notes

\newtoggle{devel}
\toggletrue{devel}
%\togglefalse{devel}

\iftoggle{devel} {
	\newcommand{\comment}[1]{{\bf \colorbox{yellow}{\color{black}#1}}}
	\newcommand{\todo}{{\bf \colorbox{red}{\color{white}TODO:}}}	
} {
	\newcommand{\comment}[1]{{}}
	\newcommand{\todo}{{}}
}

\newcommand{\ifreport}[1]{#1}
\newcommand{\ie}{{\em i.e.,~}}
\newcommand{\eg}{{\em e.g.,~}}
\newcommand{\term}[1]{\mbox{\texttt{#1}}}
\newcommand{\itl}[1]{\mbox{\textit{#1}}}

% Commas and semicolons
\newcommand{\comma}{,\,}
\newcommand{\commadots}{\comma \ldots \comma}
\newcommand{\semi}{;\mbox{;};}
\newcommand{\semidots}{\semi \ldots \semi}

% Brackets
\newcommand{\set}[1]{\{#1\}}
\newcommand{\sbs}[1]{\lquote #1 \rquote}

% Spacing
\newcommand{\gap}{\quad\quad}
\newcommand{\biggap}{\quad\quad\quad}
\newcommand{\nextline}{\\ \\}
\newcommand{\htabwidth}{0.5cm}
\newcommand{\tabwidth}{1cm}
\newcommand{\htab}{\hspace{\htabwidth}}
\newcommand{\tab}{\hspace{\tabwidth}}
\newcommand{\linesep}{\ \hrulefill \ \smallskip}

%%%%%%%%%%%%%%%%%%%%%%%%%%%%%%%%%%%%%%%%%%%%%%%%%%%%%%%%%%%%%%%%
%% Settings and tweaks

% use citep instead of cite
%\let\cite\citep
\newcommand{\sectionbreak}{\clearpage}


%Bibtex
\def\BibTeX{{\rm B\kern-.05em{\sc i\kern-.025em b}\kern-.08em
		T\kern-.1667em\lower.7ex\hbox{E}\kern-.125emX}}

%%%%%%%%%%%%%%%%%%%%%%%%%%%%%%%%%%%%%%%%%%%%%%%%%%%%%%%%%%%%%%%%
%% The following listings environments are provided:
%%
%% python     for Python code
%% xml        for XML code
%% bash       for BASH code
%%
%% This style file is based on original work by Olivier Verdier,
%% with contributions from Johan Hake.
%%
%% Modified for ANS by Anders Logg, 2011.
%% 
%% The original can be found here
%% http://www.math.chalmers.se/Math/Grundutb/CTH/tmv225/1415/dokument/rapportmall/anslistings.sty 
%%
%% First added:   2015-02-08
%% Last modified: 2015-02-08

% Basic setup
\renewcommand{\lstlistlistingname}{Code Listings}
\renewcommand{\lstlistingname}{Code Listing}
\newcommand{\codetitlestyle}[1]{\small\textit{#1}}
\newcommand{\belowtitleskip}{2pt}
\newcommand{\captionposition}{t}
\newcommand{\framemargin}{0.5ex}
\newcommand{\literatecolour}{\textcolor{literatecolour}}

% Colors
\definecolor{gray}{gray}{0.5}
\colorlet{commentcolour}{green!50!black}
\colorlet{stringcolour}{red!60!black}
\colorlet{keywordcolour}{magenta!90!black}
\colorlet{exceptioncolour}{yellow!50!red}
\colorlet{commandcolour}{blue!60!black}
\colorlet{numpycolour}{blue!60!green}
\colorlet{literatecolour}{magenta!90!black}
\colorlet{promptcolour}{green!50!black}
\colorlet{specmethodcolour}{violet}
\colorlet{indendifiercolour}{green!70!white}

%--- Typesetting Python ---

 \lstdefinestyle{pythonstyle}{
 language=python,
 showtabs=true,
 tab=,
 tabsize=2,
 basicstyle=\ttfamily\footnotesize,
 stringstyle=\color{stringcolour},
 showstringspaces=false,
 alsoletter={1234567890},
 otherkeywords={\ , \}, \{, \%, \&, \|},
 keywordstyle=\color{keywordcolour}\bfseries,
 emph={and,break,class,continue,def,yield,del,elif ,else,%
 except,exec,finally,for,from,global,if,import,in,%
 lambda,not,or,pass,print,raise,return,try,while,assert},
 emphstyle=\color{blue}\bfseries,
 emph={[2]True, False, None},
 emphstyle=[2]\color{keywordcolour},
 emph={[3]object,type,isinstance,copy,deepcopy,zip,enumerate,reversed,list,len,dict,tuple,xrange,append,execfile,real,imag,reduce,str,repr},
 emphstyle=[3]\color{commandcolour},
 emph={Exception,NameError,IndexError,SyntaxError,TypeError,ValueError,OverflowError,ZeroDivisionError},
 emphstyle=\color{exceptioncolour}\bfseries,
 commentstyle=\color{commentcolour}\slshape,
 emph={[4]ode, fsolve, sqrt, exp, sin, cos, arccos, pi,  array, norm, solve, dot, arange, , isscalar, max, sum, flatten, shape, reshape, find, any, all, abs, plot, linspace, legend, quad, polyval,polyfit, hstack, concatenate,vstack,column_stack,empty,zeros,ones,rand,vander,grid,pcolor,eig,eigs,eigvals,svd,qr,tan,det,logspace,roll,min,mean,cumsum,cumprod,diff,vectorize,lstsq,cla,eye,xlabel,ylabel,squeeze},
 emphstyle=[4]\color{numpycolour},
 emph={[5]__init__,__add__,__mul__,__div__,__sub__,__call__,__getitem__,__setitem__,__eq__,__ne__,__nonzero__,__rmul__,__radd__,__repr__,__str__,__get__,__truediv__,__pow__,__name__,__future__,__all__},
 emphstyle=[5]\color{specmethodcolour},
 emph={[6]assert,range,yield},
 emphstyle=[6]\color{keywordcolour}\bfseries,
 literate=*%
 {:}{{\literatecolour:}}{1}%
 {=}{{\literatecolour=}}{1}%
 {-}{{\literatecolour-}}{1}%
 {+}{{\literatecolour+}}{1}%
 {*}{{\literatecolour*}}{1}%
 {/}{{\literatecolour/}}{1}%
 {!}{{\literatecolour!}}{1}%
 {[}{{\literatecolour[}}{1}%
 {]}{{\literatecolour]}}{1}%
 {<}{{\literatecolour<}}{1}%
 {>}{{\literatecolour>}}{1}%
 {>>>}{{\textcolor{promptcolour}{>>>}}}{1}%
 ,%
 breaklines=true,
 breakatwhitespace= true,
 aboveskip=1ex,
 frame=trbl,
 framesep=.3ex,
 rulecolor=\color{black!40}
}

\newcommand{\inputpython}[1]{\lstinputlisting[style=pythonstyle,title={\codetitlestyle{Python code }},belowcaptionskip=\belowtitleskip]{#1}}
\lstnewenvironment{python}[1][]{\lstset{style=pythonstyle,title={\codetitlestyle{Python code }},belowcaptionskip=\belowtitleskip}}{}
\newcommand{\pyth}{\lstinline[style=pythonstyle,basicstyle=\ttfamily]}

%--- Typesetting XML ---

\lstdefinestyle{xmlstyle}{
language=xml,
showtabs=true,
tab=,
tabsize=2,
basicstyle=\ttfamily\footnotesize,
stringstyle=\color{stringcolour},
showstringspaces=false,
alsoletter={1234567890},
emphstyle=\color{exceptioncolour}\bfseries,
commentstyle=\color{commentcolour}\slshape,
breaklines=true,
breakatwhitespace= true,
aboveskip=1ex,
frame=trbl,
framesep=.3ex,
rulecolor=\color{black!40}
}

\newcommand{\inputxml}[1]{\lstinputlisting[style=xmlstyle, title={\codetitlestyle{XML code}}, belowcaptionskip=\belowtitleskip]{#1}}
\lstnewenvironment{xml}[1][]{\lstset{style=xmlstyle, title={\codetitlestyle{XML code}}, belowcaptionskip=\belowtitleskip}}{}

%--- Typesetting Bash ---

\lstdefinestyle{bashstyle}{
language=bash,
showtabs=true,
tab=,
tabsize=2,
basicstyle=\ttfamily\footnotesize,
stringstyle=\color{stringcolour},
showstringspaces=false,
alsoletter={1234567890},
otherkeywords={\ , \}, \{, \%, \&, \|},
emphstyle=\color{exceptioncolour}\bfseries,
commentstyle=\color{commentcolour}\slshape,
breaklines=true,
breakatwhitespace= true,
aboveskip=1ex,
frame=trbl,
framesep=.3ex,
rulecolor=\color{black!40}
}

\newcommand{\inputbash}[1]{\lstinputlisting[style=bashstyle, title={\codetitlestyle{Bash code}}, belowcaptionskip=\belowtitleskip]{#1}}
\lstnewenvironment{bash}[1][]{\lstset{style=bashstyle, title={\codetitlestyle{Bash code}}, belowcaptionskip=\belowtitleskip}}{}
\lstnewenvironment{csh}[1][]{\lstset{style=bashstyle, title={\codetitlestyle{Csh code}}, belowcaptionskip=\belowtitleskip}}{}